%%%%%%%%%%%%%%%%%%%%%%%%%%%%%%%%%%%%%%%%%%%%%%%%%%%%%%%%%%%%%%%%%%%%%%%%%%%%%%%%
% data_set.tex:
%%%%%%%%%%%%%%%%%%%%%%%%%%%%%%%%%%%%%%%%%%%%%%%%%%%%%%%%%%%%%%%%%%%%%%%%%%%%%%%%
\chapter{Analysis Strategy for Long-Lived Particles }
\section{Analysis Strategy}
\subsection{Signal and Background Modelling}
In modern particle physics experiments, being able to do pseudo experiments which provide reliable insight into the kinematics and detector response to the physics of interest is a necessity. CMS is no exception.
Using physics event generators like PYTHIA, Madgraph, SHERPA is very common.
The generated events are then passed into the GEANT4 which provides a simulation of the physics of the event and the response of the CMS detector.   
The resulting  event is known as a Monte Carlo Simulation ~(MC) events and the dataset is a MC dataset.
In this analysis, we have used quite a number of cathegories of MC dataset for both the signal as well as the background. The following events were produced using PYTHIA and simulated using GEANT4.
\subsection{Event Selection}
\subsubsection{Higher Level Triggering and DataSet}
\subsection{Background Estimation}
\subsection{Efficiency and Systematics Studies}


\section{Limit Setting}



%%%%%%%%%%%%%%%%%%%%%%%%%%%%%%%%%%%%%%%%%%%%%%%%%%%%%%%%%%%%%%%%%%%%%%%%%%%%%%%%
Using the excellent timing resolution as described in previous chapters,
we can use the timing information of arrival photons on the ECAL to distinguish 
between in-time photons( prompt) as well as out of time or delayed photons.
%%%%%%%%%%%%%%%%%%%%%%%%%
%%%%%%%%%%%%%%%%%%%%%%%%%%%%%%%%%%%%%%%%%%%%%%%%%%%%%%%
\label{Search_strategy_or_Analysis_chapter}