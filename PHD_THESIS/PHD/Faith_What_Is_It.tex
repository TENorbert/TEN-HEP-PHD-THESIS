 \documentclass{article}[12pt]
\usepackage{graphicx}

\begin{document}

\title{What is Faith?}% \LaTeX{}}
\author{Tambe E. Norbert}

\maketitle

\begin{abstract}
I have been struggling with emotional issues for years now. My source of being emotional is because I am a control freak! I want to control almost everything at almost every level. I cannot afford to trust the decisions of other people in a group project especially when I tend to see that their line of action of degree of involvement is not pointing as I expect towards a better solution. As a result I get emotional and want to put knowledge in them at all cost. At times I do it loud repetitively. It overwhelms me at times and I feel like I have to fight a huge battle all a lone with people not helping at all. But then I ask what if I let it all loose? I mean, what is wrong if I stop trying to execute things the way I am  most sure is the better way? What if I start looking forward and hoping for a better out come even if I am  (un)aware that their efforts seems not to be leading in the "right" direction? Why don't I believe the end product is always going to be better, whatever better means no matter what? What is faith? How do I execute faith? Must it always be contrary to what I think or expect? Can one really archive PEACE in life without faith? How necessary is faith to one's life?
\end{abstract}

\section{What is faith?}

I began studying the word of God and kept reading Matthew 6:25-34 where Jesus talks about birds and lilies in the field. I began wondering... what then is faith? I mean how close to carelessness and irresponsibility can faith be?
Then I glimpses of knowledge flow into me, I think Jesus is referring to priorities and obsessions here. The moment you begin obsessing or flood your mind in the little universe of what you see or imagine as the only solution or path way to moving or transporting you from state A in life to state B in life, you imprison yourself in this little universe of yours of whats possible and whats not possible? whats correct and whats not correct? whats worth pursuing and whats not worth pursuing? whats meaningful and whats not worth meaningful? Whats worthy and whats not worthy?
This closure or imprisonment is the very source or worry and peace thief.
You basically put yourself in a prison cell and lock the door with a key that no one else has but you alone to release you. You are trapped and caught in a logically well designed cage that no one else can understand. This is a super prison because no one else can release you and some times not even yourself as this might take a very long time. The worst of it is that, your logic is sound and flawless. So how do you come out of this well designed self imprisonment?



\begin{equation}
    \label{simple_equation}
    \textbf{PEACE} = \sqrt{\textbf{HOPE}\times \textbf{FAITH} }
\end{equation}

\subsection{How do I go about acquiring faith?}

\paragraph*{•}
First, do you even realised your are in a cell? I don't know about you but I had no idea I was locked up inside for a very long time.
Until I began asking the question after reading Matthew 6:25-34. My first question is what is faith?  I hear faith is hope of things not seen.
How the f**k does this help me? what does this mean?
Okay lets pretend I understanding this. what next? what is the boundary between irresponsibility and hope?
I mean some people say hope is does not mean do not do anything at all.. I mean that would be irresponsibility. Hope adds more to that. Do what you can and leave what you cannot to someone ( usually a mystical someone/something ) to handle the rest with the thought that the end product will be favourable.
\newline
As I meditated upon the passages I read about faith and some few google words, I began realising that, there is no way one can archive peace in life without faith. And boy, this whole chain is long.
\newline
\textbf{Faith is believing that knowledge is infinite}
and \textbf{Humility is recognising and understanding that knowledge is also infinite}
Therefore faith and humility must go hand in gloves. Possibilities towards an expected end are infinite just as knowledge is. The outcome of everything is based and driven by a particular form or type of knowledge referred commonly as an idea.
Realising that, there is always the possibility for an idea and infinite other ideas to lead to an expected end demands humility. This is the unset of faith.
The act of faith, is believing that there are always other possibilities and yes other paths to the same expected end. Are the results of these other paths towards an expected goal better or desirable? Desirable or better is based on knowledge and who knows it? Some things or expected results might seem desirable at the moment or at first but once knowledge kicks in, they are removed from the shelf of what’s desirable or better.
Therefore, no matter how equipped and knowledgeable one is... there is always room for more knowledge and one does not know everything. This is why continuous research and learning is imperative. However, the source of learning or acquiring knowledge also matters a lot but then faith comes in again.
We cannot afford to live without faith not in these crucial times after all faith makes all things possible. However, faith which makes things possible is unwavering faith. Not any type of faith.
Lord I pray for unwavering faith.


\begin{figure}
    \centering
    \includegraphics[width=3.0in]{/home/tensr/Pictures/higgs.png}
    \caption{Simulation Results}
    \label{simulationfigure}
\end{figure}

\section{Conclusion}
Write your conclusion here.

\end{document}