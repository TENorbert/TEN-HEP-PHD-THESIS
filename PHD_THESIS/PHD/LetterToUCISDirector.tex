\documentclass[12pt]{article}

\date{\today}
\begin{document}

Dear Sir,

\par
I was born in a small village called Bonadikombo, a few km from the town of Limbe, in the South West region of the Republic of Cameroon, West Africa. Cameroon has two official languages, French and English. I studied in both languages. To say that my family lived below a dollar per day would be an understatement.
My stepfather is a local bricklayer and my mother a local farmer/petite trader.
I am the eldest of five children. I succeeded in completing primary, secondary  and university school through community help and private goodwill. After graduating with a bachelor's degree in physics in 2006, I volunteered teaching  at a secondary/high school call KULU memorial college in the town of Limbe. A local public school for underprivileged children.
I taught Mathematics and Physics for a year. Earlier the following year, I was introduced to an international scholarship program sponsored by the International Atomic Energy Agency~(IAEA) and the Italian government by one of my former university lecturers. I immediately applied and was lucky enough to be selected as one of the many students from almost every less developed country in the globe. I was offered a full scholarship for pre-PhD training at the Abdus Salam International Center for Theoretical Physics~(ICTP) in Trieste, Italy. This glorious and unique institute for research and training was born in Trieste in 1964 by Pakistani-born physicist and Nobel Laureate Abdus Salam.
It is here that I fell in love with particle physics, studied for two years and earned a Double Diploma in Theoretical and High Energy Physics. These diploma programs were very intense with no breaks and continuous teaching lasting a total of 18 months and the remaining months for thesis projects. After graduating in August 2009, I was once again very lucky to get a Teaching/Research Assistant Fellowship for graduate studies at the University of Minnesota, Twin Cities where I am currently pursuing and will soon be defending my Doctor of Philosophy degree in Experimental High Energy Physics. It is during the initial years of my graduate degree that I was introduced to the Large Hadron Collider~(LHC) by one of my professors.
%\par
\paragraph*{LHC and CMS} \mbox{}\\
\par
The LHC is a particle~(protons/ions) accelerator and collider hosted by the European Organization for Nuclear Research~(CERN) in Geneva, Switzerland. CERN is a particle physics international laboratory  in which the United States like a few other nations is making  huge investments through financial and unique technical knowledge contributions for the purpose of extending and maintaining US leadership in research in fundamental science and the high energy frontier. One from many of such contributions is the design, building and maintenance of a general-purpose particle physics detector, the Compact Muon Solenoid~(CMS) detector. The purpose of this detector is to help provide scientific answers to fundamental questions about our universe and possibly beyond such as; what is our universe made of? Is our universe unique? Are there other universes out there? What is the fate of our universe? How do we get mass? Why does everything around us have mass? Is the current stuff~(material or matter) made of unique type of particles? Why is the universe over populated by matter only? What about anti-matter? what is dark matter and dark energy?.
%As you would imagine, finding answers to these questions is quite a non-trivial challenge or the responsibility of a single nation talk less of a single human. 
Scientist, Engineers, Physicists, Mathematicians form collaborations addressing these questions. The CMS is one of such many collaborations which I am involved and have made some very minor but useful contributions on behalf of the US through its CMS participation. I must take this opportunity to point that, in answering the question of what is the origin of mass and how do we get mass?, the CMS collaboration, provided an answer through the discovery of the Higgs boson on the 4th of July 2012. We now understand and can say with confidence that the Higgs boson is responsible for providing mass to all the material~(matter) we see around us and that includes us humans. My contribution  is in calibrating the electromagnetic callorimeter~(ECAL), the subdetector of the CMS detector responsible for discovering the Higgs boson. I am also contributed in the data acquisition and certification used the physics analysis of this profound discovery.

\paragraph{My Contributions}\mbox{}\\
\par
The CMS collaboration involves 4300 particle physicists, engineers, technicians, students and support staff from 182 institutes in 42 countries.
Each collaborator works in a field or section of this giant detector according to their interest. My interest is in the ECAL subdetector which is responsible for discovering the Higgs boson.
I have been the main timing calibration expert developing and implementing algorithms for the complete understanding and use of the CMS detector for measurements and search for new particles. I am also responsible for the maintenance and data acquisition by the CMS detector. The data used in a physics analysis in CMS must be certified~(accepted as useful) after undergoing  data certification approval and timing calibration approval. Articles  published by the CMS collaboration acknowledges and reference the participation and contribution of these efforts. These are the conditions to be qualified an author of article by the CMS collaboration. 
I am working on a current physics analysis for the search for new fundamental particles called \textit{neutralino} considered to be a candidate for Dark Matter particles which I have attached a front page showing the unpublished article I am currently working on. This work is in progress and will hopefully be published in a top physics journal before summer 2015. Some of research and development work are published either within the CERN publication documentation system, Nature or a top international journal like the Journal of High Energy Physics~(JHEP), European Physics Journal C for Particle and Nuclear Physics, International Journal of High Energy Physics, Elservier's Progress in Particle and Nuclear Physics and Physics Review D.  I have attached a list of such CMS published articles with my contributions from top physics article archives. The results presented in these articles are continuously being cited in articles by other authors in the physics community.

While visiting CERN in the past few years on behalf of the US-CMS contribution, I am, in addition to developing physics algorithms and writing software for detector calibration and the search for new particles, currently working on the CMS detector upgrade in preparation for the second round of proton-proton collision by the LHC collider starting March 2015. A period which I will once again be the principal detector calibration expert, data acquisition and certification expert and also performing physics analysis in search for new particles; specifically dark matter particles. The results of this physics analysis will have profound impact in the physics community as well as humanity beyond imagination.
My desire and job as a particle physicist is to use the CMS detector and LHC collider machine to understand our universe. I am afraid, in a large collaborative effort such as the CMS tackling challenging problems, an individual's ideas and work, in comparison with other scientific disciplines, might not be solely published in an article for it to be referenced by other authors as a measure of their impact but never the less are breakthroughs and reliable without which myself as an example would not be considered a contributor in all the articles published by the CMS collaboration and referenced by other authors within the physics community.
%\par
\paragraph{Conclusion}\mbox{}\\
\par
Sir, as you would have realized, the study and progress in experimental high energy physics might not be considered on equal terms with other scientific disciplines in which an individual's ideas and contributions can be easily highlighted in a very independent manner, rather our contributions towards addressing a given problem are interdependent. I believe this trend is very common in collaborative science projects in addressing more and more challenging problems of trying to understand fundamental science. 
Particle physics is unique due to the kind of problems it is addressing.
Sir, I understand, you would want me to list out published articles by other researchers highlighting my name and portion of my work being referenced showing how important my contributions to my field is. I could do this which would involved most articles in the entire field of high energy Physics citing or referencing a CMS collaboration published article using my research work and contributions with my direct or indirect contributions. However, this will not be a fair judgment as an article in experimental particle physics usually consists of a combination of individual ideas and efforts at so many different levels all entirely interdependent to each other. As a result judging an individual's impact through article referencing cannot not be performed fairly in a manner performed in many other fields. As an example, I am currently working in a physics analysis to search for new particles candidate dark matter particles. The results of this research will be published as a CMS collaboration effort and rightly so and if referenced by another research scientists or collaboration, my name would not be highlighted or cited somewhere as a form of recognition of my efforts, rather the CMS collaboration as a whole would be highlighted and referenced.  In most articles published by CMS, my name is highlighted under the University of Minnesota section similar with the CMS collaborators involved. To expand on this a bit further and as an example, here is a recent article published in Nature magazine of a combined effort by CMS and LHCb on "\textit{Observation of the rare of $B^{0}_{s} \rightarrow \mu^{+} \mu^{-}$ decay from the combined analysis of CMS and LHCb data}" has the following statement: 
\newline
\textit{ \textbf{Author Contributions}: All authors have contributed to the publication, being variously involved in the design and the construction of the detectors, in writing software, calibrating sub-systems, operating the detectors and acquiring data and finally analyzing the processed data. 
\newline
\textbf{Author Information}:Correspondence and requests for materials should
be addressed to \textbf{cms-publication-committee-chair@cern.ch} and to
\textbf{lhcb-editorial-board-chair@cern.ch}}. 
\newline
\par
It is my hope that your judgment would warrant me favorable in my Immigrant Petition for Alien Worker as a look forward to an exciting future working in particle physics. \newline The United States of America has a unique history and culture of bold and relentless leadership in humanity's quest for knowledge. It is this notion and culture that I want to make contributions and defend.
%It is clear that I am starting young but it is even clearer that the future is up for grasp and I want to be claiming that future.
\newline

Thank you.

Tambe E. Norbert


\paragraph{Statement of Qualification} \mbox{}\\
\paragraph{Substantial Intrinsic Merit} \mbox{} \\
I seek Immigration Petition for Notional Interest Waiver for Alien Worker
in the field of Sub-Atomic Physics and its applications. I want to professionally teach Physics in a university while simultaneously doing research and development in the application of sub-atomic physics in the developing renewable energy, nuclear energy and devices, nuclear health services and marketable technology development. 
I believe this field to have substantial national interest to the US in the sections of health, safety, economy, environment and education.

\paragraph{National in Scope} \mbox{}\\
I am currently a PhD candidate working in the field of sub-atomic or Particle Physics~( or Nuclear Physics) trying to understand fundamental particles and their interactions. The knowledge derived from my current PhD research will improve in developing new materials and energy sources for application in renewable energy, technology advancement through sensor device development, health care; through cancer therapy, diagnostic instrumentation in medical imaging  and nuclear energy and security through monitoring nuclear waste nonproliferation. The current research is carried out in the the LHC is a particle~(protons/ions) accelerator and collider hosted by the European Organization for Nuclear Research~(CERN) in Geneva, Switzerland. We use the  Compact Muon Solenoid~(CMS) detector, a general-purpose particle detector in which the US is a major sponsoring partner to study these fundamental particles and their interactions. All funding and cooperation is provided by the US department of Energy under High Energy Particle physics research.

\paragraph{National Interest} \mbox{}\\
Understanding fundamental particle interaction requires performing physics analysis using data acquired using the CMS detector at CERN during LHC proton or ion collisions. The LHC will begin colliding particles again in March 2015.
As a visiting scientists to CERN on behalf of the US-CMS contribution, I plan to continue in addition to developing physics algorithms and writing software for detector calibration and the search for new particles, working on the CMS detector upgrade in preparation for the second round of proton-proton collision by the LHC collider starting March 2015. As evidence, I have attached along with other documents, the renewal of my contract of association with CERN for the period of November, 18th 2014 to November, 17th 2015. During this time period, I will once again be the principal detector calibration expert, data acquisition and certification expert and also performing physics analysis in search for new particles; specifically dark matter particles. The results of this physics analysis will have profound impact in the physics community and the material science. Which in addition to bolstering US leadership in fundamental science will also expand on our knowledge for developing new materials and tools to advance technology applicable in the health, nuclear safety, economic and environmental sectors. An evidence for this is seen in the numerous papers published by the CMS collaboration in top physics journals whose results comprising of my ideas and work are continuously being cited and useful in particle and nuclear physics by the whole physics and engineering community. The spin projects from my work with at CERN could lead to developing new materials for radiation protection which can easily be useful in Cancer treatment therapy as well as sending humans to Mars. The algorithms and software developed can also be applied in data science for analyzing huge amounts of data in the business, health and government sectors which will improve the economy. This is a currently very rapidly growing field known as Big Data Science.


\end{document}