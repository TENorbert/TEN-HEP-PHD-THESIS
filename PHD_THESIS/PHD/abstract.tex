%%%%%%%%%%%%%%%%%%%%%%%%%%%%%%%%%%%%%%%%%%%%%%%%%%%%%%%%%%%%%%%%%%%%%%%%%%%%%%%%
% abstract.tex: Abstract This is the Abstract Section of Analysis.
%%%%%%%%%%%%%%%%%%%%%%%%%%%%%%%%%%%%%%%%%%%%%%%%%%%%%%%%%%%%%%%%%%%%%%%%%%%%%%%%

%%%%%%%%%%%%%%%%%%%%%%%%%%%%%%%%%%%%%%%%%%%%%%%%%%%%%%%%%%%%%%%%%%%%%%%%%%%%%%%%
Dark matter particles, if they exists in nature, are believed to be neutral, stable, weakly interacting with ordinary matter and maybe massive. The hunt for dark matter particles is on! There are theoretical models which predict the  existence of dark matter particles that can be produced in a proton-proton collider like the Large Hadron Collider, with sufficient center of mass energy. Gauge Mediating Supersymmetric Model is an example of such models. It describes the production and decay into isolated energetic photons, new, massive, neutral, weakly interacting particles known as supersymmetric particles. The lightest supersymmetric particle, the Neutralino~($\tilde{\chi}^{0}_{1}$), is a prime example and its decay into a photon is accompanied by a light weakly interacting and stable particle, the gravitino~($\tilde{G}$). The properties of the gravitino makes it a very good candidate for dark matter particles. The resulting photon from such a decay, is understood to be delayed in its arrival time at the detector due to inherent dynamics understood to be well described by supersymmetry and not the current theory, the standard model. The signature of a delayed photon is not specific to only supersymmetric models but could be the result of probably some new model not relating to the standard model. Using the Compact Muon Solenoid detector at the LHC, we have performed a search for delayed photons produced from proton-proton collisions at the center of mass energy, $\sqrt{S} = 8$~TeV. We did not find any excess of events over standard model background. Consequently, we compute an upper limit on the possible existence of a lightest neutralino with  mass and proper lifetime; $m_{\tilde{\chi}^{0}_{1}} \geq \chi\chi$~$GeV/c^{2}$ and $\tau_{\tilde{\chi}^{0}_{1}} \geq \chi\chi$~ns respectively, as described in the gauge mediated supersymmetric models. We also show that using only the timing information of the electromagnetic calorimeter as an observable, the  CMS detector is sensitive to neutralinos with life time up to $30$~ns which no previous experiment had shown. We provide hints on possible improvements which might help discover delayed photons in future search analysis.