%%%%%%%%%%%%%%%%%%%%%%%%%%%%%%%%%%%%%%%%%%%%%%%%%%%%%%%%%%%%%%%%%%%%%%%%%%%%%%%%
% abstract.tex: Abstract This is the Abstract Section of Analysis.
%%%%%%%%%%%%%%%%%%%%%%%%%%%%%%%%%%%%%%%%%%%%%%%%%%%%%%%%%%%%%%%%%%%%%%%%%%%%%%%%

%%%%%%%%%%%%%%%%%%%%%%%%%%%%%%%%%%%%%%%%%%%%%%%%%%%%%%%%%%%%%%%%%%%%%%%%%%%%%%%%
Dark matter particles, if they exists in nature, are believed to be neutral, weakly interacting with ordinary matter and at times massive. The hunt for dark matter particles is on! There are theoretical models which predict the  existence of dark matter particles that can be produced in a proton-proton collider with sufficient center of mass energy like the Large Hadron Collider. One of such models, is the Gauge mediating super symmetric models which describes the production and decay into isolated energetic photons of supersymmetric particles which are massive, neutral and weakly interacting like the lightest Neutralino~($\tilde{\chi}^{0}_{1}$). The resulting photon from such a decay, is understood to be delayed in its arrival time at the detector due to inherent dynamics only understood to be well described within supersymmetry  and not by the current theory the standard model. The signature of a delayed photon is not only specific to supersymmetric models but can be as a result of probably some new model ofcourse not currently related to the standard model. Using the compact muon solenoid detector at the LHC, we have searched for delayed photons produced from proton-proton collisions at the center of mass energy, $\sqrt{S} = 8$~TeV. We did not find any excess of events over standard model background events. As a result, we have set an upper limit on the possible existence of a lightest neutralino with  mass and proper lifetime; $m_{\tilde{\chi}^{0}_{1}} \geq XX$~$GeV/c^{2}$ and $\tau_{\tilde{\chi}^{0}_{1}} \geq XX$~ns respectively as described in the gauge mediated supersymmetric models. We also show that using only the timing information of the electromagnetic calorimeter as an observable, the  CMS detector is sensitive to neutralinos whose life time is up to $30$~ns which no previous experiment could show. We provide hints on possible improvements which might help discover delayed photons in future search analysis.