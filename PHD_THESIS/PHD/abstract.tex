%%%%%%%%%%%%%%%%%%%%%%%%%%%%%%%%%%%%%%%%%%%%%%%%%%%%%%%%%%%%%%%%%%%%%%%%%%%%%%%%
% abstract.tex: Abstract This is the Abstract Section of Analysis.
%%%%%%%%%%%%%%%%%%%%%%%%%%%%%%%%%%%%%%%%%%%%%%%%%%%%%%%%%%%%%%%%%%%%%%%%%%%%%%%%

%%%%%%%%%%%%%%%%%%%%%%%%%%%%%%%%%%%%%%%%%%%%%%%%%%%%%%%%%%%%%%%%%%%%%%%%%%%%%%%%
%Dark matter particles are believed to be neutral, stable, weakly interacting with ordinary matter and maybe massive. The hunt for dark matter particles is on! There are theoretical models which predict the  existence of dark matter particles that can be produced in a proton-proton collider like the Large Hadron Collider provided there is sufficient center of mass energy. Gauge Mediating Supersymmetric Models are examples of such models. These models describe the production and decay into isolated energetic photons, new, massive, neutral, weakly interacting particles known as supersymmetric particles. The Neutralino~($\tilde{\chi}^{0}_{1}$), being the lightest supersymmetric particle in terms of mass, is a prime example and its decay into a photon is accompanied by a light weakly interacting and stable supersymmetric particle, the gravitino~($\tilde{G}$) which is considered to be a very good candidate for dark matter particles. The resulting photon from such a decay, is understood to be delayed in its arrival time at the detector. This timing delay is due to inherent dynamics understood and well described by models beyond the standard model. The signature of a delayed photon is not specific to only supersymmetric models but could be the result of probably some new kind of physics unrelated to the standard model. Using the Compact Muon Solenoid detector at the LHC, 
We performed a search for delayed photons produced during proton-proton collisions with center of mass energy, $\sqrt{S} = 8$~\TeV. In the absence of excess events over standard model prediction, we produce limits on the cross section, $\sigma_{\PSneutralinoOne} > 0.02$~pb, for the production and decay of the lightest neutralino, $\PSneutralinoOne$, with  mass, $m_{\PSneutralinoOne} \geq 235$~\GeVcc, and lifetime, $\tau_{\PSneutralinoOne} \ge 35$~ns, as described in supersymmetry. We also show that using only timing information of the CMS electromagnetic calorimeter as observable, the  CMS detector is sensitive to neutralino with lifetime up to $30$~ns and mass, $m_{\PSneutralinoOne} \approx 260$\GeVcc. %We provide possible improvements  of our search analysis which might help discover delayed photons in future.