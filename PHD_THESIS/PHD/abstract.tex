%%%%%%%%%%%%%%%%%%%%%%%%%%%%%%%%%%%%%%%%%%%%%%%%%%%%%%%%%%%%%%%%%%%%%%%%%%%%%%%%
% abstract.tex: Abstract This is the Abstract Section of Analysis.
%%%%%%%%%%%%%%%%%%%%%%%%%%%%%%%%%%%%%%%%%%%%%%%%%%%%%%%%%%%%%%%%%%%%%%%%%%%%%%%%

%%%%%%%%%%%%%%%%%%%%%%%%%%%%%%%%%%%%%%%%%%%%%%%%%%%%%%%%%%%%%%%%%%%%%%%%%%%%%%%%
We perform a search for long-lived neutral particles in final state with delayed photons and large missing transverse energy produced in LHC proton-proton collisions at center-of-mass energy, $\sqrt{S} = 8$\TeV. Capitalizing on the excellent timing resolution of the CMS Electromagnetic Calorimeter, the search uses photon time measurements made by the Electromagnetic Calorimeter as the main search quantity. We found a single event consistent with our background expectations from the Standard Model and set a model-independent upper limit of 4.37 on number of signal events. We also interpret our results in the context of the SPS8 benchmark GMSB model and show that neutralinos with  mean lifetime, $\tau_{\PSneutralinoOne} \leq 45$\ns, and mass, $m_{\PSneutralinoOne} \leq 300$\GeVcc, or effective Supersymmetry breaking energy scale, $\mathbf{\Lambda} \leq 220\TeV$, are ruled out of existence at 95\% $CL_{S}$ confidence level. The exclusion limit on the product of the production cross-section and branching ratio of the neutralino to photon and gravitino decay channel,  $\sigma_{(\PSneutralinoOne \rightarrow \gamma + \tilde{G})}\times BR$, for different lifetimes and masses is derived. Our results confirm for the first time that the CMS Electromagnetic Calorimeter provides good sensitivity to search for long-lived neutral particles with lifetime up to $40$\ns and masses up to $300$\GeVcc using only timing measurements. %A discussion of the Electromagnetic Calorimeter time  measurement and performance is also presented.
%%%%%%%%%%%%%%%%%%%%%%%%%%%%%%%%%%%%%%%%%%%%%%%%%%%%%%%%%%%%%%%%%%%%%%%%%%%%%%%%%%%%%%%%%%%%%%%%%%%
%We provide possible improvements  of our search analysis which might help discover delayed photons in future.%Dark matter particles are believed to be neutral, stable, weakly interacting with ordinary matter and maybe massive. The hunt for dark matter particles is on! There are theoretical models which predict the  existence of dark matter particles that can be produced in a proton-proton collider like the Large Hadron Collider provided there is sufficient center of mass energy. Gauge Mediating Supersymmetric Models are examples of such models. These models describe the production and decay into isolated energetic photons, new, massive, neutral, weakly interacting particles known as supersymmetric particles. The Neutralino~($\tilde{\chi}^{0}_{1}$), being the lightest supersymmetric particle in terms of mass, is a prime example and its decay into a photon is accompanied by a light weakly interacting and stable supersymmetric particle, the gravitino~($\tilde{G}$) which is considered to be a very good candidate for dark matter particles. The resulting photon from such a decay, is understood to be delayed in its arrival time at the detector. This timing delay is due to inherent dynamics understood and well described by models beyond the standard model. The signature of a delayed photon is not specific to only supersymmetric models but could be the result of probably some new kind of physics unrelated to the standard model. Using the Compact Muon Solenoid detector at the LHC, 