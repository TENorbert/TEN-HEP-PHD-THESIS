%%%%%%%%%%%%%%%%%%%%%%%%%%%%%%%%%%%%%%%%%%%%%%%%%%%%%%%%%%%%%%%%%%%%%%%%%%%%%%%%
% conclusion.tex:
%%%%%%%%%%%%%%%%%%%%%%%%%%%%%%%%%%%%%%%%%%%%%%%%%%%%%%%%%%%%%%%%%%%%%%%%%%%%%%%%
\chapter{Conclusion}
\label{conclusion_chapter}
%%%%%%%%%%%%%%%%%%%%%%%%%%%%%%%%%%%%%%%%%%%%%%%%%%%%%%%%%%%%%%%%%%%%%%%%%%%%%%%%
We perform a search for delayed photons produced from the decay of the lightest neutralino which is a Long-Lived Neutral Particle~(LLNP) using  the arrival time of the photon measured by the ECAL sub-detector of the CMS detector. In the absence of any significant excess of signal events over the Standard Model predicted background events, we proceed with the interpretation of the results of our search  by evaluating exclusion limits  on the mass of the lightest neutralino or effective supersymmetry~(SUSY) breaking scale and lifetime of the lightest neutralino according the SPS8 benchmark GBSM model. 
\newline
We showed that light neutralinos with  mass, $m_{\PSneutralinoOne} \leq 300$\GeV/cc or effective SUSY breaking scale, $\mathbf{\Lambda} \leq 220\TeV$, and $\tau_{\PSneutralinoOne} \leq 45$~ns are ruled out of existence at 95\% Confidence Limit~(CL) computed using the $CL_{s}$ method from a search on 19.1\fbinv integrated luminosity of the 2012 $8$\TeV LHC $pp$ collision data recorded by the CMS detector. This corresponds to an upper limit on the production cross section times branching ratio of the photon and gravitino decay channel,  $\sigma^{UP}_{\PSneutralinoOne} \geq 0.01$~pb . 
\newline
We produced 95\% Confidence Limit~(CL) on the lifetime, mass of lightest neutralino or effective SUSY breaking scale and the production cross section times branching ratio.
This search produced the most strongest limits produced so far and show for the first time that using only ECAL timing measurements, the CMS detector has a really food sensitivity and is reliable in the search for LLNP decaying to electromagnetic particles. Hopefully with the increase luminosity and center of mass energy of 13\TeV, during the LHC Run 2, we may find such a LLNP using the CMS detector.
%%%%%%%%%%%%%%%%%%%%%%%%%%%%%%%%%%%%%%%%%%%%%%%
%In this model the LLNP is the lightest neutralino, \PSneutralinoOne, which is the next-to-lightest SUSY particle and decays into the lightest supersymmetric particle, the gravitino~($\tilde{G}$), which is a candidate dark matter particle, and a photon with a finite lifetime.
%%%%%%%%%%%%%%%%%%%%%%%%%%%%%%%%%%%%%%%%%%%%%%%%%%%%%%%%%%%%%%%%%%%%%%%%%%%%%%%%
