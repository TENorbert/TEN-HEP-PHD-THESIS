%%%%%%%%%%%%%%%%%%%%%%%%%%%%%%%%%%%%%%%%%%%%%%%%%%%%%%%%%%%%%%%%%%%%%%%%%%%%%%%%
% conclusion.tex:
%%%%%%%%%%%%%%%%%%%%%%%%%%%%%%%%%%%%%%%%%%%%%%%%%%%%%%%%%%%%%%%%%%%%%%%%%%%%%%%%
\chapter{Conclusion}
\label{conclusion_chapter}
%%%%%%%%%%%%%%%%%%%%%%%%%%%%%%%%%%%%%%%%%%%%%%%%%%%%%%%%%%%%%%%%%%%%%%%%%%%%%%%%
We have performed a search analysis for NMLLP decaying to photons using the time of arrival of the photon as measured by the ECAL sub detector of the CMS detector.
Haven fail to find any significant signal of delayed photons over the standard model background, we interpreted our results in SUSY models with NMLLP like SPS8 of minimal GBSM or general GMSB models.
We showed that, neutralinos whose production and decay mechanism is described in the SPS8 mGMSB model,  with $m_{\tilde{\chi}^{0}_{1}} \leq XX$~$GeV/c^{2}$ and $\tau_{\tilde{\chi}^{0}_{1}} \leq XX$~ns are ruled out of existence at 95\% confidence level using the 2012 $8$~TeV LHC dataset. This corresponds to an upper limit of  $\sigma^{UP}_{\tilde{\chi}^{0}_{1}} \geq XX$~pb  on the production cross section times branching ratio in a hadron collider. 
In addition, we mention some of the limitations in this particular analysis from a detector point of view and how in future studies can be improved.
We hope that in the future, with increase in center of mass energy of the LHC collider as well as luminosity and an improve in timing resolution beyond what is already very reliable, we will surely find a new fundamental particle whose dynamics cannot be described by the already very successful standard model of particle physics.


%%%%%%%%%%%%%%%%%%%%%%%%%%%%%%%%%%%%%%%%%%%%%%%%%%%%%%%%%%%%%%%%%%%%%%%%%%%%%%%%
