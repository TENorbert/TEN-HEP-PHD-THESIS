%%%%%%%%%%%%%%%%%%%%%%%%%%%%%%%%%%%%%%%%%%%%%%%%%%%%%%%%%%%%%%%%%%%%%%%%%%%%%%%%
% conclusion.tex:
%%%%%%%%%%%%%%%%%%%%%%%%%%%%%%%%%%%%%%%%%%%%%%%%%%%%%%%%%%%%%%%%%%%%%%%%%%%%%%%%
\chapter{Conclusion}
\label{conclusion_chapter}
%%%%%%%%%%%%%%%%%%%%%%%%%%%%%%%%%%%%%%%%%%%%%%%%%%%%%%%%%%%%%%%%%%%%%%%%%%%%%%%%
We perform a search for delayed photons which may be produced from the decay of a Long-Lived Neutral Particle~(LLNP) like the neutralino by measuring the arrival time of the photon with the CMS Electromagnetic Calorimeter~(ECAL). This search was done on data which is equivalent to 19.1\fbinv of integrated luminosity produced from LHC proton-proton collisions at a center-of-mass energy of $8$\TeV and recorded by the CMS detector. We set an upper limit on the production cross-section of a LLNP. 


\begin{comment}
\newline
Our exclusion limits on the lifetime, mass of lightest neutralino or effective SUSY breaking scale and the production cross section times branching ratio are the strongest limits using timing measurements alone to search for events with delayed photon. This is the first time that the CMS ECAL timing has been used as the only search quantity to search for LLNPs and the results demonstrate that CMS ECAL timing has good sensitivity and provides a reference for measuring future improvement on the sensitivity from additional search quantities, particularly in search experiments which use a combination different search quantities. 
\end{comment}
%Hopefully with the increase luminosity and center of mass energy of 13\TeV, during the LHC Run 2, we may find such a LLNP using the CMS detector.
%%%%%%%%%%%%%%%%%%%%%%%%%%%%%%%%%%%%%%%%%%%%%%%%%%%%%%%%%%%%%%%%%%%%%%%%%%%%%%%%%%%%%%%%%%%%%%%%%%%%%%%%%
%In this model the LLNP is the lightest neutralino, \PSneutralinoOne, which is the next-to-lightest SUSY particle and decays into the lightest supersymmetric particle, the gravitino~($\tilde{G}$), which is a candidate dark matter particle, and a photon with a finite lifetime.
%%%%%%%%%%%%%%%%%%%%%%%%%%%%%%%%%%%%%%%%%%%%%%%%%%%
%%Exclusion limits  on the mass of the lightest neutralino or effective supersymmetry~(SUSY) breaking scale and lifetime of the lightest neutralino according the SPS8 benchmark GBSM model show light neutralinos with  mass, $m_{\PSneutralinoOne} \leq 300$\GeV/cc or effective SUSY breaking scale, $\mathbf{\Lambda} \leq 220\TeV$, and $\tau_{\PSneutralinoOne} \leq 45$~ns are ruled out of existence at 95\% Confidence Limit~(CL) computed using the $CL_{s}$ method. This corresponds to an upper limit on the production cross section times branching ratio of the photon and gravitino decay channel,  $\sigma^{UP}_{\PSneutralinoOne} \geq 0.01$~pb . 
%%%%%%%%%%%%%%%%%%%%%%%%%%%%%%%%%%%%%%%%%%%%%%%%%%%%%%%%%%%%%%%%%%%%%%%%%%%%%%%%
