%%%%%%%%%%%%%%%%%%%%%%%%%%%%%%%%%%%%%%%%%%%%%%%%%%%%%%%%%%%%%%%%%%%%%%%%%%%%%%%%
% conclusion.tex:
%%%%%%%%%%%%%%%%%%%%%%%%%%%%%%%%%%%%%%%%%%%%%%%%%%%%%%%%%%%%%%%%%%%%%%%%%%%%%%%%
\chapter{Conclusion}
\label{conclusion_chapter}
%%%%%%%%%%%%%%%%%%%%%%%%%%%%%%%%%%%%%%%%%%%%%%%%%%%%%%%%%%%%%%%%%%%%%%%%%%%%%%%%
We have performed a search analysis for NMLLP decaying to photons using the time of arrival of the photon as measured by the ECAL subdetector of the CMS detector.
We did not find any signal of delayed photons and as such decided to interpret our results in the context of GBSM. We also discuss some of the weakness of our analysis from a detector point of view as well future studies  which can be done to improve on the search strategy and analysis.
We hope that in the future, with increase in center of mass energy of the LHC collider as well as luminosity and an improve in timing resolution beyond what is already very reliable, we will surely find a new fundamental physics particle beyond what is already known in the current standard model.


%%%%%%%%%%%%%%%%%%%%%%%%%%%%%%%%%%%%%%%%%%%%%%%%%%%%%%%%%%%%%%%%%%%%%%%%%%%%%%%%
