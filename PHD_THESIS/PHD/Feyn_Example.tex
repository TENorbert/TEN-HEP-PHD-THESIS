\documentclass{article}
\usepackage[utf8]{inputenc}
\usepackage{feynmf}

\begin{document}

\begin{fmffile}{diagram}

Here is a simple Feynman diagram:

\vspace{1em} % Not important, just adds some space

\begin{fmfgraph*}(120,80)
    \fmfleft{i1,i2}
    \fmfright{o1,o2}
    \fmf{fermion}{i1,v1,o1}
    \fmf{fermion}{i2,v2,o2}
    \fmf{photon}{v1,v2}
\end{fmfgraph*}

\vspace{1em}

Here is the same diagram with some labels:

\vspace{1em}

\begin{fmfgraph*}(120,80)
    \fmfleft{i1,i2}
    \fmfright{o1,o2}
    \fmf{fermion}{i1,v1,o1}
    \fmf{fermion}{i2,v2,o2}
    \fmf{photon}{v1,v2}
    \fmflabel{$v_1$}{v1}
    \fmflabel{$v_2$}{v2}
\end{fmfgraph*}

\vspace{1em}

Here are some more complicated examples: % from http://szczypka.web.cern.ch/szczypka/guides/latex/feynmp.html

\vspace{1em}

\begin{fmfgraph*}(100,100)
    \fmfleft{i1}
    \fmfright{o1,o2}
    \fmf{fermion,label=$u$}{i1,w1}
    \fmf{fermion,label=$d$}{w1,o1}
    \fmf{photon,label=$W^{+}$}{w1,o2}
    \fmfv{lab=$V^{\ast}_{ud}$,lab.dist=0.05w}{w1}
\end{fmfgraph*}
    
\begin{fmfgraph*}(200,200)
    % bottom and top verticies
    \fmfstraight
    \fmfleft{i0,i1,i2,id1,id2,i3,i4,i5}
    \fmfright{o0,o1,o2,od1,od2,o3,o4,o5}
    % incoming proton to gluon vertices
    \fmf{fermion,label=$d$}{i1,o1}
    % tension shifts vertex to one side
    \fmf{fermion,tension=1.5,label=$\overline{b}$}{v2,i4}
    \fmf{fermion,label=$\overline{c}$}{o4,v2}
    \fmffreeze
    \fmf{fermion}{o2,v3,o3}
    \fmf{fermion,label=$\overline{s}$}{o2,v3}
    \fmf{fermion,label=$c$}{v3,o3}
    \fmf{photon, tension=2,label=$W^{+}$}{v2,v3}
    % phantom centres the W->cs vertex
    \fmf{phantom,tension=1.5}{i1,v3}

    \fmfv{lab=$V_{cb}^{\ast}$}{v2}
    \fmfv{lab=$V_{cs}$,lab.dist=-.1w}{v3}
\end{fmfgraph*}

\begin{fmfgraph*}(200,200)
    %bottom and top verticies
    \fmfbottom{P1,P2}
    \fmftop{P1',b,bbar,P2'}
    %incoming protons to gluon vertices
    \fmf{fermion,tension=2,lab=$P_1$}{P1,g1}
    \fmf{fermion,tension=2,lab=$P_2$}{P2,g2}
    %blobs at gluon vertices, 0.16w is the size of blob
    \fmfblob{.16w}{g1,g2}
    %gluon from P1 to vertex1
    \fmf{gluon,lab.side=right,lab=$x_{1}P_{1}$}{g1,v1}
    %gluon from P2 to vertex2 - note change of order!
    \fmf{gluon,lab.side=right,lab=$x_{2}P_{2}$}{v2,g2}
    %quark loop was here
    \fmf{fermion, tension=.6, lab.side=right,lab=$b$}{v1,b}
    \fmf{fermion, tension=1.2}{v2,v1}
    \fmf{fermion, tension=.6, lab.side=right,lab=$\overline{b}$}{bbar,v2}
    %outgoing protons
    \fmf{fermion}{g1,P1'}
    \fmf{fermion}{g2,P2'}
    %freeze everything in place
    \fmffreeze
    \renewcommand{\P}[3]{\fmfi{plain}{%
        vpath(__#1,__#2) shifted (thick*(#3))}}
    %lines on P1
    \P{P1}{g1}{2,0}
    \P{P1}{g1}{-2,1}
    %lines on p2
    \P{P2}{g2}{2,1}
    \P{P2}{g2}{-2,0}
    %lines on P1'
    \P{g1}{P1'}{-2,-1}
    \P{g1}{P1'}{2,0}
    %lines on P2'
    \P{g2}{P2'}{-2,0}
    \P{g2}{P2'}{2,-1}
\end{fmfgraph*}

\end{fmffile}

\end{document}
