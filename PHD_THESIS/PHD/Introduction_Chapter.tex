%%%%%%%%%%%%%%%%%%%%%%%%%%%%%%%%%%%%%%%%%%%%%%%%%%%%%%%%%%%%%%%%%%%%%%%%%%%%%%%
% intro.tex: Introduction to the thesis
%%%%%%%%%%%%%%%%%%%%%%%%%%%%%%%%%%%%%%%%%%%%%%%%%%%%%%%%%%%%%%%%%%%%%%%%%%%%%%%%
\chapter{Introduction}
\label{intro_chapter}
%%%%%%%%%%%%%%%%%%%%%%%%%%%%%%%%%%%%%%%%%%%%%%%%%%%%%%%%%%%%%%%%%%%%%%%%%%%%%%%%
%The main style and messaging of my thesis will look like this:
%\begin{enumerate}
%\item Introduction as to why would anyone care about neutral long lived particles.
%\item What is/are the mechanisms behind a long lived particle? can you give an example and show this mechanism at work explicitly?
%\item How are searches for these particles performed and why use the LHC-CMS machines? eg Is there a particular reason why this is the case? Can these experiment be done using other machines?
%\item Have this searches been performed before? If yes, what are the results? What are their limits?  Can you beat their limits? What makes you think you can? What advantages or edge do you have over previous results?
%\item  How is the experiment performed? What specific issue(s) about the LHC-CMS machine makes it an ideal instrument?
%\item  What results did you get? Are they better than previous results? if not why? Can one improve on the results? What is the future of this kind of
%search? Can it be performed again  in future? What  advantages do future experiments have over this current experiments?
%\item  What are your thoughts(conclusion) about the whole process?
%\item  Where does one go from here?
%\end{enumerate}

%In this Thesis, I am going to write about three main aspects.
%\begin{enumerate}
%\item  Why long-lived particles  and the mechanism behind its possible existence?.
%\item  How to use the CMS detector to search for a long-lived particle.
%\item How is the search done?
%\item  How do I set an upper limit on the possible existence of a long-lived particle in case I fail to observe it?
%\item  What is the future for long-lived particle with the increase in LHC center of mass energy and luminosity?
%\end{enumerate}

\paragraph*{} \mbox{}\\
 %model-independent
We have performed a search for Neutral Massive Long-Lived Particles~(NMLLP) decaying to photons using timing information. This analysis uses data recorded using the CMS detector from proton-proton~(\textbf{$pp$}) collisions by  the Large Hadron Collider~(LHC) with a center of mass energy  $\sqrt{S} = 8 $ TeV.
\mbox{}\\
Particles from the decay of NMLLP could be \textit{Dark Matter}~(DM) particles as they are weakly interacting and stable.
Matter observed around us make up only 4.5\% of our total universe or multiverse. It is best described with unmatched precision by simple symmetries known as gauge symmetries. The mathematical formulation of gauge symmetries used in our understanding of the visible or baryonic universe is best implemented in the Standard Model~(SM). The SM cannot describe non-visible or non-baryonic matter~(DM) which make up the larger percentage of matter content in our universe. Although a direct detection of DM is yet to be presented, indirect detection experiments in Cosmology and Astrophysics support speculation that non-visible matter is made up of particles which may be very stable or have long lifetime collectively called \textit{Long-Lived}~(LL) particles.  In general, LL particles are either charged~(electromagnetically charged~(i.e interact with light~(photons)) or color charged) or neutral~(i.e cannot interact with light in the context of the SM).
\newline
Of particular interest to the scientific community are neutral LL particles, since DM is understood to not interact directly with light and could very weakly interact with visible matter.  Recent negative search results is indicating that dark matter particles, if they exists, could be very light i.e having very small mass of about a few eV to keV. These are known as Warm Dark Matter~(WDM). DM particles could also be heavy with mass in the \GeV to \TeV mass range. These are called Cold Dark Matter~(CDM). A common  property is that they are stable.
\paragraph*{}\mbox{}\\
The phenomenon of interest is a delayed photon produced in the decay of a meta-stable Next-to-Lightest Supersymmetric Particle~(NLSP). The NLSP is the NMLLP. A classic example of a NLSP is the neutralino~($\tilde{\chi}^{0}_{1} $). It decays into a photon and the Lightest Supersymmetric Particle~(LSP) called the gravitino~($\tilde{G}$). In R-parity conserving~(RPC) models, supersymmetric particles like the neutralino are produced in pairs at a  particle collider. The neutralinos are produced in a cascade decay of higher massive  supersymmetric particles produced from proton-proton~($pp$) collisions. The gravitino being the LSP is stable, light in mass, neutral and weakly interacting with ordinary matter. This makes it a good candidate particle for DM. The photons from neutralino decay are energetic, isolated and delayed in their arrival time at a detector. These photons can be detected using the electromagnetic calorimeter. High transverse momentum~(\pt) spray of hadronic particles collectively called jets and missing transverse energy due to the weakly interacting nature of the gravitino as it leaves the detector undetected, accompany this decay. The measured photon arrival time at the electromagnetic calorimeter  is large~(nanoseconds~(ns)) because of the inherently long neutralino lifetime and extra distance traveled inside the detector. This combination of jets, missing transverse momentum and delayed photon is a clear signal for a new kind of physics beyond the standard model~(SM).
An event with the decay of a neutralino, produced in the LHC $pp$ collider would be recorded using the Compact Muon Solenoid~(CMS) detector.
The CMS detector is located at one of the beam crossing or collision points~(also known as Interaction Point~(IP)) at Point 5 in Cessy, France.
Relying on the excellent timing and energy resolution of Electromagnetic Calorimeter~(ECAL) sub detector, of the CMS detector, we can distinguish between high energy photons from NMLLP decay and photons
produced in interactions precisely and well described by the SM. 
Finding such a LL particle would address a lot of important questions in modern physics like: Why do we observe so much matter than anti-matter in our universe?
Is there a reason why particles as currently observed in the SM have very different masses and can be classified to exists in 3 generations? What is the origin and existence of Dark Matter~(DM) and what is it made of?  
Do all fundamental forces behave the same way at some higher energy scale?
Answers to these questions will provide a clear understanding and direction towards studying physics beyond the standard model.
\paragraph*{} \mbox{}\\
We have described in this thesis, a search analysis for delayed photons with results. This description is arranged as follows in the following chapters: 
\begin{itemize}
%science or Motivation
\item Chapter 1 gives an introduction and general outline of this thesis. 

% Models and Phenomenology
\item In Chapter 2, we give a brief description of the current standard model highlighting its strengths and weaknesses. We also describe \textit{supersymmetric} models which are theoretical extensions beyond the standard model~(BSM) embedded with the prediction of NMLLP.
The physics of long-lived particle is also described.
This chapter also presents compelling hints from theory, experiment as well as cosmological observation supporting the existence of NMLLPs which motivates our search.
The phenomenology of NMLLP in \textit{gauge mediating supersymmetric} models is used as a benchmark model in our search. Results from previous search analysis are also presented. 

%experiment or LHC & CMS
\item  Chapter 3,  describes the experimental setup particularly the LHC collider and CMS detector giving detail description of its sub-detectors emphasizing on those which have been used in our search analysis.

%ECAL timing reconstruction and Calibration
\item Timing reconstruction and calibration is described in chapter 4, detailing the method of extraction and calibration procedure used by CMS.  
%Physics object reconstruction and Identification in CMS
\item In chapter 5, the reconstruction of physics objects such as superclusters, photons, jets and missing transverse energy~(\MET) according the CMS standards is described here. The presence of anomalous signals in the electromagnetic calorimeter is also mentioned.

% Search Analysis for LL particles
\item The search analysis is described in chapter 6 detailing triggers used, dataset, choice of observables, event selection and background estimation techniques used. The result of the search is also presented here. Sources of systematic and quantification considered in this experiment are also described.

%Statistical Analysis
\item Chapter 7 presents the statistical analysis and methods used providing clear meaning of $p$-values as used in this search analysis.
%hashes out the strategy for analysis and presents the data and simulated sets that will be used in the analysis.

%Limit Interpretation & Future
\item Using the minimal \textit{Gauge Mediating Supersymetric Model}~(mGMSB) with \textit{Snowmass Signal Point} 8~(SPS8) as our benchmark model, chapter 8 provides an interpretation of our results in terms of exclusion regions reached by our analysis. Possible improvement for future analysis is briefly mentioned. 

%Conclusion.
\item  In chapter 9, we present our conclusion from performing the search for delayed photons.

\end{itemize}


%\textbf{Some comments from Mississippi Snowmass 2013 I like.}
%\paragraph*{Comments BSM?: Convener: Marcela Carena}
%Energy Frontier
%comments: Kyle Cranmar:
%Questions: 
%Comments: 
%1)Measuring couplings of Higgs now @ LHC before ILC provide Full width.
%2)No clear guidance at the moment to Empirical measurements reliable the most for now.
%3) Smaller and shorter  time scale experiments better at the moment.
%Comments: Andrei de Gouvan
%1) Neutrino masses do not sit well in SM
%2) Dark matter also not SM

%Where to find new physics?
%1) Neutrino, Dark Matter ? not easy
%2) New Sources of CP -Invariance violation (EDMs) sensitive to very high energy scales
%3)  Searches for violations
%4) Precision measurements for Higgs Parameters and very rare processes?
%5)  Explore LHC

%Comments : Jonathan Feng:
%1)  Neutrino Mass/theory of  Flavor - Neutrino Experiments
%2) Higgs Boson and Naturalness( gauge Hierarchy problem- precision measurements of Higgs Boson parameters
%3) Dark Matter -Experiments and search
%4) Dark Energy and Modified gravity- Cosmological problem.
%5) Matter Asymmetry - CP violations can be source to find Matter Asymmetry
%6)  Inflation - Evidence for New physics can be found using -Experiments of CMB

%Comments: Nima-Akarni Amed
%1) Neutrino Oscillations + DM
%2) Conservative Ideas still required to describe nature /observed data?
%3) Effective Field theory of relativistic nature can also describe Condensed Matter physics, However, never has
%a Scalar field been used as seen in the case of Higgs.
%4) New Physics at TeV Scale making naturalness stable
%5) Higgs Physics is fine-tuned thus needs precision measurements.
%6) Is New physics as a result of naturalness or Not? - Test Naturalness with HL-LHC
%7) HL-LHC killing naturalness? unlikely
%8) If at O(100TeV) no New physics then fundamental shift paradigm in thinking
%Expts: Proton decay,  DM sources, EDMs,  Build O(100 TeV) Machines

%1) Explore unnatural areas to search for new physics by designing new Tools
%2) Look back at clues to gain new Ideas for new measurements
%3) Exploit existing facilities but also build new facilities.
%4) Explore new territories-Using Neutrino Beams, Dark Matter, precision measurements
%5) Explore Old ideas in the light of new technologies.
%6) Redo already performed experiments using new technologies, data taking and Analyzing technologies.
%Recent paper of Meta-stable particles
%arXiv:hep-ph/0908.0315

%Question? Why is the top quark mass so different from others: 
%Answer in SUSY?.
%some CMS Bulletin Papers
%Some impt CMS/ECAL papers to read.
%1)CMSBUL-ARTICLE-2013-005
%2) CMSBUL-ARTICLE-2013-023 %(http://cds.cern.ch/record/1561186?ln=en)
%3) CMSBUL-ARTICLE-2013-012%(http://cds.cern.ch/record/1544496?ln=en)
%4) ECAL Timing Prospects %(http://cms.cern.ch/iCMS/jsp/db_notes/showNoteDetails.jsp?noteID=CMS-20CR-2013/186)
%5) 2013 ECAL Performancs NOTE: CMS DP -2013/016, CMS DP -2013/007 


%%%%%%%%%%%%%%%%%%%%%%%%%%%%%%%%%%%%%%%%%%%%%%%%%%%%%%%%%%%%%%%%%%%%%%%%%%%%%%%%
