%%%%%%%%%%%%%%%%%%%%%%%%%%%%%%%%%%%%%%%%%%%%%%%%%%%%%%%%%%%%%%%%%%%%%%%%%%%%%%%
% intro.tex: Introduction to the thesis
%%%%%%%%%%%%%%%%%%%%%%%%%%%%%%%%%%%%%%%%%%%%%%%%%%%%%%%%%%%%%%%%%%%%%%%%%%%%%%%%
\chapter{Introduction}
%%%%%%%%%%%%%%%%%%%%%%%%%%%%%%%%%%%%%%%%%%%%%%%%%%%%%%%%%%%%%%%%%%%%%%%%%%%%%%%%
%The main style and messaging of my thesis will look like this:
%\begin{enumerate}
%\item Introduction as to why would anyone care about neutral long lived particles.
%\item What is/are the mechanisms behind a long lived particle? can you give an example and show this mechanism at work explicitly?
%\item How are searches for these particles performed and why use the LHC-CMS machines? eg Is there a particular reason why this is the case? Can these experiment be done using other machines?
%\item Have this searches been performed before? If yes, what are the results? What are their limits?  Can you beat their limits? What makes you think you can? What advantages or edge do you have over previous results?
%\item  How is the experiment performed? What specific issue(s) about the LHC-CMS machine makes it an ideal instrument?
%\item  What results did you get? Are they better than previous results? if not why? Can one improve on the results? What is the future of this kind of
%search? Can it be performed again  in future? What  advantages do future experiments have over this current experiments?
%\item  What are your thoughts(conclusion) about the whole process?
%\item  Where does one go from here?
%\end{enumerate}
%In this Thesis, I am going to write about three main aspects.
%\begin{enumerate}
%\item  Why long-lived particles  and the mechanism behind its possible existence?.
%\item  How to use the CMS detector to search for a long-lived particle.
%\item How is the search done?
%\item  How do I set an upper limit on the possible existence of a long-lived particle in case I fail to observe it?
%\item  What is the future for long-lived particle with the increase in LHC center of mass energy and luminosity?
%\end{enumerate}
%%\paragraph*{} \mbox{}\\
\par
  Astronomical observations such as gravitational lensing, galaxy rotation curves and precision cosmology studies through the Wilkinson Microwave Anisotropy Prope~(WMAP), all indicate the presence of a new form of matter in the universe which interacts significantly with visible matter through gravity only \cite{DM}. This illusive matter in the universe, known as \textit{Dark Matter}~(DM) makes up about 23\% of the universe and does not emit or scatter off electromagnetic radiation. Direct detection of  DM is yet to be reported and its supposed existence is one of the strongest indications for a new kind of physics beyond the Standard Model~(SM) of particle physics, as there are no known particles in the SM which can be attributed to DM. 
\par
The SM is a theoretical model formulated based on fundamental symmetries of nature called \textit{gauge symmetries}. Based on these gauge symmetries, the SM provides all our understanding of the fundamental composition and their interactions~(which include the weak, electromagnetic and strong interactions, ordered in increasing interaction strength) of visible matter. Visible matter does emit and can scatter off electromagnetic radiation and make up only 4.5\% of the total matter in the universe. All the predictions and calculations made by the SM have been experimental verified to unmatched precision. Yet, the SM fails to describe gravitational interactions and DM which are both responsible for most of the matter structure observed in our universe. 
\newline
Theoretical models which are extensions of the standard model allow for the existence of new fundamental symmetries and new fundamental and composite particles around a 1000\GeV energy scale which is accessible by some hadron colliders. These models are called \textit{Beyond Standard Models}~(BSM). Among the many BSMs, \textit{Supersymmetry}~(SUSY) is the most studied.  SUSY extends the gauge symmetries of the SM to a much larger family of symmetries and allows for the doubling of the particles in the SM to include a wide variety of new particles which could be dark matter particles\cite{SUSYDM,LSPDM}. A particular family of supersymmetry models called Gauge Mediating Supersymmetry Breaking~(GMSB) models, predict the production, at a particle collider like the LHC, new, massive, neutral, weakly interacting and long-lived particles which can decay into a candidate dark matter particle and a delayed energetic photon. For simplicity, we will refer to such new, massive, weakly interacting and neutral long-lived particles which can decay into a photon and candidate dark matter particle as a \textit{Neutral Massive Long-Lived Particle}~(NMLLP).
\newline
Our current understanding of DM particles comes from results from theoretical models and numerical N-body simulations \cite{DMS}, which support speculation that dark matter particles are neutral and stable \ie particles which can remain stable for long periods of time comparable to the age of the universe, before disintegrating. DM particles could themselves be produced from the disintegration~(decay) of not very stable or meta-stable particles with sufficiently long lifetimes to travel the size of a particle detector before disintegrating. These meta-stable particles  could be these NMLLPs. There are effort directed towards the search for dark matter particles produced from the decay of NMLLPs..%\cite{LSPDM}. %Experimental techniques for detecting neutral particles are very limited and so previous methods for searching for DM particles are often being updated with few new methods entirely developed.  %Nevertheless, the ultimate goal is to detect dark matter particles!
%We believe DM consists of particles which are neutral, might be stable or meta-stable, very weakly interacting with ordinary matter and maybe massive.
%%\paragraph*{}%\mbox{}\\
\par
   The search for dark matter particles cover a wide range of experiments, from deep space search experiments like the Hubble and James Webb Space Telescope experiments, the Alpha Magnetic Spectrometer detector on board the International Space Station of NASA, ground based particle detector experiments like the Super Cryogenic Dark Matter Search experiment which try to detect dark matter particles produced in cosmic rays, to collider experiments like the LHC of the European National Laboratory for Nuclear Research in Geneva, Switzerland, where dark matter particles can be produced during particle collisions. The interest in collider experiments is base on theoretical model predictions that dark matter particles or particles which decay into dark matter particles may be produced at the LHC, provided there is sufficient center-of-mass energy.
The candidate dark matter particle produced is stable, neutral and does not interact with the particle detector material. It also has a lifetime comparable to the age of the universe.  Such a DM particle will be observed through Missing Transverse Energy~(MET) at the LHC in a particle detector.
\newline
Detecting delayed photons  and  measuring MET, because of the undetected DM particle, from the decay of a supersymmetric NMLLP will be an indication of the presence of new physics interactions BSM, as there are no neutral, weakly interacting and long-lived particles which decay into delayed photons and MET in the SM. Numerous previous experiments have searched for either delayed or prompt photons and MET.
The results from these experiments have not been positive and depends on a particular search method  used. New search techniques which depend on the production and decay mechanism of the NMLLP are being developed. %Nevertheless, new experiments with new detection techniques and clever search methods depending on the manner of production and decay of these new particles are continuously being developed.  
%Of particular interest to the scientific community are neutral LL particles, since DM is understood to not interact directly with light and could very weakly interact with visible matter.  Recent negative search results is indicating that dark matter particles, if they exists, could be very light i.e having very small mass of about a few eV to keV. These are known as Warm Dark Matter~(WDM). DM particles could also be heavy with mass in the \GeV to \TeV mass range. These are called Cold Dark Matter~(CDM). A common  property is that they are stable.
\par %\mbox{}\\
This thesis involves the search for a delayed photon produced from the decay of a NMLLP which in our case is the Next-To-Lightest Supersymmetric Particle~(NLSP) called the \textit{lightest neutralino}~(\PSneutralinoOne). The \PSneutralinoOne decays into a photon and the Lightest Supersymmetric Particle~(LSP) called the gravitino~($\tilde{G}$) which is our DM particle as described in \textit{R-Parity Conserving}~(RPC) GMSB models\cite{KOlive}, like the \textit{Snowmass Signal Point} 8~(SPS8) which is our benchmark model. The lightest neutralino is produced from proton-proton collisions at the LHC and its decay products can be detected using the general purpose Compact Muon Solenoid~(CMS) particle detector. The arrival time of the photon is measured relying on the excellent timing resolution of the Electromagnetic Calorimeter~(ECAL).
 %We consider the NMLLP to be the. This lightest neutralino decays into .  The gravitino is neutral and weakly interacting with visible matter, and being the LSP makes it stable. These properties makes the gravitino a good dark matter particle.. The neutralino decay can in general be instantaneous~(prompt) or \textit{delayed} depending on the choice of parameters in the neutralino decay model. In a subset of models called , sypersymmetry particles are pair produced either, directly during particle collisions, or from the cascade decay of higher massive supersymmetry particles produced also in particle collisions. As a result, in these models, the LSP is very stable and automatically a dark matter candidate particle while the NLSP~(neutralino) is often long-lived and the photons are energetic, isolated and often delayed in their arrival time at a detector. These often delayed photons can be detected using the electromagnetic calorimeter~(ECAL) of the CMS detector.
\par
The ECAL has an excellent timing resolution better than a \textit{nanosecond}~(ns) in addition to its superior energy resolution. It is made of  75,848 \pb crystals arranged in a barrel~(61,200 crystals) and endcap~(14648 crystals) geometries. The crystals detect photons and electrons through scintillation and in the process measure their energy and arrival time with good precision. The scintillation light from the crystals is collected using photo-detectors attached to the crystals and converts the light into electrical signals which is readout using radiation hard readout electronic application specific integrated  chips.
Due to the high and constant radiation dose in the LHC, the crystal's have to be time calibrated at least once a month and their radiation damaged constantly monitored using lasers to ensure optimal performance throughout the entire LHC proton-proton collisions. 
\newline
The crystals are time calibrated using photons produced from proton-proton collisions and their timing performance studied using $\PZ \rightarrow \EE$ boson decay events. We described in later chapters, how the ECAL timing calibration is performed to obtain and maintain its excellent timing resolution.
\newline
Finding an event with a delayed photon event at the LHC using ECAL timing measurements, will confirm the existence of NMLLPs and will help answer important questions in particle physics like: What is the source and nature of dark matter?, Is there any reason why known SM particles are classified into 3 generations and have very different masses as known? Why do we observe so much matter compared to anti-matter in our universe? Is there a single or many universes?
Answers to these questions will provide a clear understanding and direction for future research in physics beyond the SM.
%%In cases where the neutralino is produced from the cascade decay of higher massive particles, large transverse momentum~(\pt) spray of hadronic particles collectively called jets are also part of the neutralino production and decay event. Since the gravitino is weakly interacting with the detector material, it is undetected. The gravitino presence is inferred using \textit{missing transverse momentum} which in combination with the transverse momenti of the jets and photons should conserve the total momentum of the event in the transverse plane of the detector.
%In the decay scenario of the neutralino where its lifetime is large, say above, $3$~ns, the photon is delayed and its measured arrival time is large~(many nanoseconds~(ns)). This is because of the inherently long neutralino lifetime and the extra distance it has to travel inside the detector before is decays. 
%%Finding an event, with the combination of jets, missing transverse momentum and at least a delayed photon is a clear signal for new particled not known in the standard model~(SM).
%%The CMS detector is located in the LHC tunnel at one of the proton bunch crossing points also known as \textit{Interaction Point}~(IP).
%%We measure the arrival time of the photon from the IP to the surface of the ECAL. Relying on the excellent timing and energy resolution of the ECAL, we can distinguish between high energy photons from the decay of a NMLLP from those produced by the SM which are often prompt and not delayed. To ensure this excellent timing resolution, the ECAL detector is continuously time calibrated throughout the entire LHC proton-proton collision year.%to allow for optimum performance in its timing measurements. 
%% We will describe in future chapters how  the ECAL timing alignment is done to realize this excellent timing resolution.
%We have described the search for a NMLLP decaying to a photon using arrival time information of the photon to the ECAL and use this information to distinguish between the signal from new long-lived particles and background from SM interactions. Our search analysis uses data recorded using the CMS detector produced from proton-proton~(\textbf{$pp$}) collisions at the Large Hadron Collider~(LHC) with a center-of-mass energy  $\sqrt{S} = 8 $\TeV.
\par %\mbox{}\\
The content of this thesis is arranged starting with this introduction as chapter 1, followed by chapter 2, which gives a 
brief description of the SM, highlighting its strengths and weaknesses and motivating the need to go beyond the SM in our efforts towards understanding the universe. The study of SUSY specifically  GMSB models which allows for the existence of NMLLP that can decay into a delayed photon and a gravitino is also presented.
In Chapter 3,  we describe the  LHC and CMS particle detector, dueling only on the sub-detectors which have been used in our search analysis. How the ECAL measures the arrival time of a particle is described in chapter 4.
Event reconstruction along with its constituent particles is described in chapter 5 with the definition of quantities like jets and MET~(\MET) according the CMS standards is described. Anomalous signals called \textit{spikes} observed in the  ECAL is briefly described.
% Search Analysis for LL particles 
Our search method is described in chapter 6 including details of the data samples and triggers we used, our trigger efficiency, search observable, event selection and background estimation techniques used. We also describe  in this chapter the various systematic sources considered and their contribution to our search result.
%Our search result ed here. Sources of systematic and quantification considered in this experiment are also described.
%Statistical Analysis
Chapter 7 presents the statistical and analysis methods used with clear meaning of $p$-values as used in our analysis.
%hashes out the strategy for analysis and presents the data and simulated sets that will be used in the analysis.
%Limit Interpretation & Future
The cross section times branching ratio limits depending on the lifetime and mass of the production and decay of a NMLLP in the minimal GMSB models and \textit{Snowmass Signal Point} 8~(SPS8) as our benchmark model is presented in chapter 8. The interpretation of our results in terms of exclusion regions covered by our analysis is also presented in this chapter. %Possible improvement for future 
Chapter 9 covers the conclusion.



%\textbf{Some comments from Mississippi Snowmass 2013 I like.}
%\paragraph*{Comments BSM?: Convener: Marcela Carena}
%Energy Frontier
%comments: Kyle Cranmar:
%Questions: 
%Comments: 
%1)Measuring couplings of Higgs now @ LHC before ILC provide Full width.
%2)No clear guidance at the moment to Empirical measurements reliable the most for now.
%3) Smaller and shorter  time scale experiments better at the moment.
%Comments: Andrei de Gouvan
%1) Neutrino masses do not sit well in SM
%2) Dark matter also not SM

%Where to find new physics?
%1) Neutrino, Dark Matter ? not easy
%2) New Sources of CP -Invariance violation (EDMs) sensitive to very high energy scales
%3)  Searches for violations
%4) Precision measurements for Higgs Parameters and very rare processes?
%5)  Explore LHC

%Comments : Jonathan Feng:
%1)  Neutrino Mass/theory of  Flavor - Neutrino Experiments
%2) Higgs Boson and Naturalness( gauge Hierarchy problem- precision measurements of Higgs Boson parameters
%3) Dark Matter -Experiments and search
%4) Dark Energy and Modified gravity- Cosmological problem.
%5) Matter Asymmetry - CP violations can be source to find Matter Asymmetry
%6)  Inflation - Evidence for New physics can be found using -Experiments of CMB

%Comments: Nima-Akarni Amed
%1) Neutrino Oscillations + DM
%2) Conservative Ideas still required to describe nature /observed data?
%3) Effective Field theory of relativistic nature can also describe Condensed Matter physics, However, never has
%a Scalar field been used as seen in the case of Higgs.
%4) New Physics at TeV Scale making naturalness stable
%5) Higgs Physics is fine-tuned thus needs precision measurements.
%6) Is New physics as a result of naturalness or Not? - Test Naturalness with HL-LHC
%7) HL-LHC killing naturalness? unlikely
%8) If at O(100TeV) no New physics then fundamental shift paradigm in thinking
%Expts: Proton decay,  DM sources, EDMs,  Build O(100 TeV) Machines

%1) Explore unnatural areas to search for new physics by designing new Tools
%2) Look back at clues to gain new Ideas for new measurements
%3) Exploit existing facilities but also build new facilities.
%4) Explore new territories-Using Neutrino Beams, Dark Matter, precision measurements
%5) Explore Old ideas in the light of new technologies.
%6) Redo already performed experiments using new technologies, data taking and Analyzing technologies.
%Recent paper of Meta-stable particles
%arXiv:hep-ph/0908.0315

%Question? Why is the top quark mass so different from others: 
%Answer in SUSY?.
%some CMS Bulletin Papers
%Some impt CMS/ECAL papers to read.
%1)CMSBUL-ARTICLE-2013-005
%2) CMSBUL-ARTICLE-2013-023 %(http://cds.cern.ch/record/1561186?ln=en)
%3) CMSBUL-ARTICLE-2013-012%(http://cds.cern.ch/record/1544496?ln=en)
%4) ECAL Timing Prospects %(http://cms.cern.ch/iCMS/jsp/db_notes/showNoteDetails.jsp?noteID=CMS-20CR-2013/186)
%5) 2013 ECAL Performancs NOTE: CMS DP -2013/016, CMS DP -2013/007 
%%%%%%%%%%%%%%%%%%%%%%%%%%%%%%%%%%%%%%%%%%%%%%%%%%%%%%%%%%%%%%%%%%%%%%%%%%%%%%%%
\label{intro_chapter}
%%%%%%%%%%%%%%%%%%%%%%%%%%%%%%%%%%%%%%%%%%%%%%%%%%%%%%%%%%%%%%%%%%%%%%%%%%%%%%%%