%%%%%%%%%%%%%%%%%%%%%%%%%%%%%%%%%%%%%%%%%%%%%%%%%%%%%%%%%%%%%%%%%%%%%%%%%%%%%%%
% intro.tex: Introduction to the thesis
%%%%%%%%%%%%%%%%%%%%%%%%%%%%%%%%%%%%%%%%%%%%%%%%%%%%%%%%%%%%%%%%%%%%%%%%%%%%%%%%
\chapter{Introduction}
\label{intro_chapter}
%%%%%%%%%%%%%%%%%%%%%%%%%%%%%%%%%%%%%%%%%%%%%%%%%%%%%%%%%%%%%%%%%%%%%%%%%%%%%%%%
%The main style and messaging of my thesis will look like this:
%\begin{enumerate}
%\item Introduction as to why would anyone care about neutral long lived particles.
%\item What is/are the mechanisms behind a long lived particle? can you give an example and show this mechanism at work explicitly?
%\item How are searches for these particles performed and why use the LHC-CMS machines? eg Is there a particular reason why this is the case? Can these experiment be done using other machines?
%\item Have this searches been performed before? If yes, what are the results? What are their limits?  Can you beat their limits? What makes you think you can? What advantages or edge do you have over previous results?
%\item  How is the experiment performed? What specific issue(s) about the LHC-CMS machine makes it an ideal instrument?
%\item  What results did you get? Are they better than previous results? if not why? Can one improve on the results? What is the future of this kind of
%search? Can it be performed again  in future? What  advantages do future experiments have over this current experiments?
%\item  What are your thoughts(conclusion) about the whole process?
%\item  Where does one go from here?
%\end{enumerate}

%In this Thesis, I am going to write about three main aspects.
%\begin{enumerate}
%\item  Why long-lived particles  and the mechanism behind its possible existence?.
%\item  How to use the CMS detector to search for a long-lived particle.
%\item How is the search done?
%\item  How do I set an upper limit on the possible existence of a long-lived particle in case I fail to observe it?
%\item  What is the future for long-lived particle with the increase in LHC center of mass energy and luminosity?
%\end{enumerate}

\paragraph*{} \mbox{}\\
 Matter as observed around us makes up only 4.5\% of our total universe or multiverse as some currently believe. This matter is best described with unmatched precision by simple symmetries known as gauge symmetries. The mathematics of gauge symmetry currently used in our understanding of the visible or baryonic universe is best implemented in the Standard Model~(SM). The SM cannot describe non-visible or non-baryonic matter which make up the larger percentage of matter content in our universe. Observational experiments in Cosmology and Astrophysics  tend to reinforce speculation that this non-visible or \textit{Dark Matter}~(DM) could consist of particles which have long lifetime collectively called \textit{Long-Lived}~(LL) particles.  In general, LL particles can be charged~(electromagnetically charged i.e interact with light~(photons) or color charged) or neutral~(cannot interact with light in the context of the SM).
\newline
Of particular interest to the scientific community are neutral LL particles, since DM is understood to not have a direct interaction with light and could very weakly interact with visible matter.  Recent results seems to be indicating that dark matter particles, if they exists, could be very light i.e having very small mass around eV to keV mass range. These are known as Warm Dark Matter~(WDM). DM particles could also be heavy with mass around \GeV to \TeV mass range called Cold Dark Matter~(CDM). A peculiar property is that they are stable.
\paragraph*{}
In this thesis we have performed a model-independent search for  Neutral Massive Long Lived Particles~(NMLLP) 
decaying to photons. NMLLP can be produced in proton-proton~(\textbf{$pp$}) collisions by  the Large Hadron Collider~(LHC) 
with a center of mass energy  $\sqrt{S} = 8 $ TeV. 
The phenomenon of interest are delayed photons produced in the decay of a metastable next-to-lightest supersymmetric particle~(NLSP). The NLSP is this case is the NMLLP. A classic example of a NLSP is the neutralino~($\tilde{\chi}^{0}_{1} $). It decays into a photon and the lightest supersymmetric particle~(LSP) called the gravitino~($\tilde{G}$). Assuming the conservation of a symmetry known as R-parity, the neutralinos are pair produced in a cascade decay of higher massive  supersymmetric particles produced from pp collision. While the gravitino being the LSP is stable and together with it being neutral and weakly interacting makes it a good candidate particle for DM. The resulting energetic and delayed photon, isolated in this case, is detected using the electromagnetic calorimeter. High transverse momentum~(\pt) spray of hadronic particles called jets and missing transverse energy due to the weakly interacting nature of the gravitino as it leaves the detector undetected, accompany this decay. The photon arrival time at the electromagnetic calorimeter is delayed due to the long lifetime of the neutralino. 
An event containing the decay of a neutralino, produced in the LHC $pp$ collider with be recorded using the Compact Muon Solenoid~(CMS) detector.
The CMS detector is located at one of the beam crossing or collision points~(also known as Interacting Point~(IP)) at Point 5 in Cessy, France.
Relying on the excellent timing and energy resolution of Electromagnetic Calorimeter~(ECAL) sub detector, of the CMS detector, we can distinguish between high energy photons from NMLLP decay and photons
produced in interactions excellently and precisely described by the Standard Model (SM). 
Finding a LL particle would address a lot of important questions in modern physics such as the following:
Why is there so much matter and not anti-matter around us?
Is there an explanation for the particle content and mass hierarchy as currently observed in the SM? 
What is the origin and existence of Dark Matter~(DM)?  
Do all fundamental forces behave as a single force at some higher energy scale?
Answers to these questions will provide a clear path for physics beyond the standard model.
\paragraph*{}
This thesis is divided into the following chapters:
\begin{itemize}
%science or Motivation
\item Chapter 1 presents the introduction and general outline of this thesis. 

\item In Chapter 2, we begin with the motivation behind a search for neutral long lived particles predicted to exist is some BSM.
The physics of long-lived particle is also described.
This chapter also presents compelling  hints from theory, experiment as well as cosmological observation which point to the possible existence of NMLLPs.
An example BSM model used as guiding model to this analysis is also described leading to our event selection based of its predicted phenomenology. 
Previous analysis and their results are also shown. 

%experiment or LHC & CMS
\item  In Chapter 3,  describes the experimental setup particularly the LHC collider and CMS detectors and also a
detail description of the sub-detectors of the CMS which have been used in our search analysis.
%briefly presents the history of, and science behind, the subjects presented in this thesis.

%simulation or ECAL Timing Calibration
\item In Chapter 4, we provide a detail outline of Time measurement and Calibration of ECAL in order to 
reliably use timing as tool to search for long-lived particles which decay into photons and electrons depositing their energy unto the crystals of the ECAL.
%the experiment is outlined.
%reconstruction or Particle ID using ECAL Timing
\item Chapter 5,
The reconstruction of objects referred to as photons, jets and missing transverse energy ~\MET according the CMS standards is described here. The ECAL timing as useful for particle identification is presented.
%describes the simulation process used in the analysis.

% data_set  or Analysis Search Strategy
\item Chapter 6  hashes out our search strategy for neutral massive long-live particles. 
%follows the chain of reconstruction software used to obtain meaningful results from data.

%event_selection or Interpretation of results in GMSB
\item Chapter 7 provides an interpretation of the results obtained by our analysis  in the context of Gauge Mediated Supersymmetry Breaking(GMSB).
%hashes out the strategy for analysis and presents the data and simulated sets that will be used in the analysis.

%analysis or Future work
\item Chapter 8 outlines the possibility of future analysis strategy as we prepare to begin running LHC fully at $\sqrt{S} = 13$ \TeV in 2015. 
%7 demonstrates the implementation of the event selection processes.

%\item In Chapter 8 those events selected in Chapter 7 are analyzed.

%conclusion.
\item  In chapter 9, we conclude by giving a summary, presenting our results and possible improvement in this of the analyses.

\end{itemize}


%\textbf{Some comments from Mississippi Snowmass 2013 I like.}
%\paragraph*{Comments BSM?: Convener: Marcela Carena}
%Energy Frontier
%comments: Kyle Cranmar:
%Questions: 
%Comments: 
%1)Measuring couplings of Higgs now @ LHC before ILC provide Full width.
%2)No clear guidance at the moment to Empirical measurements reliable the most for now.
%3) Smaller and shorter  time scale experiments better at the moment.
%Comments: Andrei de Gouvan
%1) Neutrino masses do not sit well in SM
%2) Dark matter also not SM

%Where to find new physics?
%1) Neutrino, Dark Matter ? not easy
%2) New Sources of CP -Invariance violation (EDMs) sensitive to very high energy scales
%3)  Searches for violations
%4) Precision measurements for Higgs Parameters and very rare processes?
%5)  Explore LHC

%Comments : Jonathan Feng:
%1)  Neutrino Mass/theory of  Flavor - Neutrino Experiments
%2) Higgs Boson and Naturalness( gauge Hierarchy problem- precision measurements of Higgs Boson parameters
%3) Dark Matter -Experiments and search
%4) Dark Energy and Modified gravity- Cosmological problem.
%5) Matter Asymmetry - CP violations can be source to find Matter Asymmetry
%6)  Inflation - Evidence for New physics can be found using -Experiments of CMB

%Comments: Nima-Akarni Amed
%1) Neutrino Oscillations + DM
%2) Conservative Ideas still required to describe nature /observed data?
%3) Effective Field theory of relativistic nature can also describe Condensed Matter physics, However, never has
%a Scalar field been used as seen in the case of Higgs.
%4) New Physics at TeV Scale making naturalness stable
%5) Higgs Physics is fine-tuned thus needs precision measurements.
%6) Is New physics as a result of naturalness or Not? - Test Naturalness with HL-LHC
%7) HL-LHC killing naturalness? unlikely
%8) If at O(100TeV) no New physics then fundamental shift paradigm in thinking
%Expts: Proton decay,  DM sources, EDMs,  Build O(100 TeV) Machines

%1) Explore unnatural areas to search for new physics by designing new Tools
%2) Look back at clues to gain new Ideas for new measurements
%3) Exploit existing facilities but also build new facilities.
%4) Explore new territories-Using Neutrino Beams, Dark Matter, precision measurements
%5) Explore Old ideas in the light of new technologies.
%6) Redo already performed experiments using new technologies, data taking and Analyzing technologies.
%Recent paper of Meta-stable particles
%arXiv:hep-ph/0908.0315

%Question? Why is the top quark mass so different from others: 
%Answer in SUSY?.
%some CMS Bulletin Papers
%Some impt CMS/ECAL papers to read.
%1)CMSBUL-ARTICLE-2013-005
%2) CMSBUL-ARTICLE-2013-023 %(http://cds.cern.ch/record/1561186?ln=en)
%3) CMSBUL-ARTICLE-2013-012%(http://cds.cern.ch/record/1544496?ln=en)
%4) ECAL Timing Prospects %(http://cms.cern.ch/iCMS/jsp/db_notes/showNoteDetails.jsp?noteID=CMS-20CR-2013/186)
%5) 2013 ECAL Performancs NOTE: CMS DP -2013/016, CMS DP -2013/007 


%%%%%%%%%%%%%%%%%%%%%%%%%%%%%%%%%%%%%%%%%%%%%%%%%%%%%%%%%%%%%%%%%%%%%%%%%%%%%%%%
