%%%%%%%%%%%%%%%%%%%%%%%%%%%%%%%%%%%%%%%%%%%%%%%%%%%%%%%%%%%%%%%%%%%%%%%%%%%%%%%
% intro.tex: Introduction to the thesis
%%%%%%%%%%%%%%%%%%%%%%%%%%%%%%%%%%%%%%%%%%%%%%%%%%%%%%%%%%%%%%%%%%%%%%%%%%%%%%%%
\chapter{Introduction}
\label{intro_chapter}
%%%%%%%%%%%%%%%%%%%%%%%%%%%%%%%%%%%%%%%%%%%%%%%%%%%%%%%%%%%%%%%%%%%%%%%%%%%%%%%%
The main style and messaging of my thesis will look like this:
\begin{enumerate}
\item Introduction as to why would anyone care about neutral long lived particles.
\item What is/are the mechanisms behind a long lived particle? can you give an example and show this mechanism at work explicitly?
\item How are searches for these particles performed and why use the LHC-CMS machines? eg Is there a particular reason why this is the case? Can these experiment be done using other machines?
\item Have this searches been performed before? If yes, what are the results? What are their limits?  Can you beat their limits? What makes you think you can? What advantages or edge do you have over previous results?
\item  How is the experiment performed? What specific issue(s) about the LHC-CMS machine makes it an ideal instrument?
\item  What results did you get? Are they better than previous results? if not why? Can one improve on the results? What is the future of this kind of
search? Can it be performed again  in future? What  advantages do future experiments have over this current experiments?
\item  What are your thoughts(conclusion) about the whole process?
\item  Where does one go from here?
\end{enumerate}

In this Thesis, I am going to write about three main aspects.
\begin{enumerate}
\item  Why long-lived particles  and the mechanism behind its possible existence?.
\item  How to use the CMS detector to search for a long-lived particle.
\item How is the search done?
\item  How do I set an upper limit on the possible existence of a long-lived particle in case I fail to observe it?
\item  What is the future for long-lived particle with the increase in LHC center of mass energy and luminosity?
\end{enumerate}

\paragraph*{} 
Our current understanding of the properties of fundamental particles which make up matter in the universe is accurately described
by the Standard Model~(SM). However, this model only describes visible or baryonic matter. Non-visible or non-baryonic matter
which make up a larger percentage of matter content of our universe currently has very little understanding.  Information
concerning the age of the universe fuel speculation that Non-visible or Dark Matter~(DM) as it is referred to presently must be 
made up of particles which have long lifetime known as Long Lived~(LL) particles.  In general, LL particles can be charged~(electromagnetically charged i.e interact with light~(photons) or color charged) as well as neutral.
\newline
Of particular interest to the scientific community are neutral LL particles; since DM is also currently understood to not interact with photons  directly or interacts very weakly with visible matter.  Recent results seems to be indicating that these particles  if they exists could have very small mass around eV to keV mass range refered to as Warm Dark Matter (WDM) or heavy with mass around \GeV to \TeV mass range called Cold Dark Matter (CDM).
\paragraph*{}
In this thesis we have performed a model-independent search for  Neutral Massive Long Lived Particles~(NMLLP) 
decaying to photons. NMLLP are produced in \textbf{pp} collisions by  the Large Hadron Collider~(LHC) 
with a center of mass energy  $\sqrt{S} = 8 $ TeV. 
The phenomenon we search for is that of delayed photons which can be produced from the decay of metastable next-to-lightest supersymmetric particle ~( NLSP acting as our NMLLP ) into a light gravitino which in this case is the lightest supersymmetric particle~(LSP). Assuming a conservation of R-parity, the neutralinos are pair produced in a cascade decay of higher massive  supersymmetric particles produced from pp collision. The resulting energetic and delayed or displaced isolated photons are detected in association with high transverse momentum spray of hadronic particles called jets as well as missing transverse energy. 
The photons and other particles produced are recorded using 
the Compact Muon Solenoid (CMS) detector.
The CMS detector is located at one of the beam crossing or collision points ( also known as Interacting Points (IP) ) at Point 5 in Cessy, France.
We depend on the excellent timing and energy resolution of Electromagnetic Calorimeter
(ECAL) subdetector of the CMS to distinguish between high energy photons from NMLLP decay and normal photons
produced in interactions excellently and precisely described by the Standard Model (SM). 
Finding a Long-Lived~(LL) particle would address a lot of important questions in modern physics such as the following:
Why is there so much matter and not anti-matter around us?
Is there an explanation for the particle content and mass hierarchy as currently observed in the SM? 
What is the origin and existence of Dark Matter~(DM)?  
Do all fundamental forces behave as a single force at some higher energy scale?
Answers to these questions will provide a clear path for physics beyond the standard model.
\paragraph*{}
This thesis is divided into the following chapters:
\begin{itemize}
%science or Motivation
\item Chapter 1 presents the introduction and general outline of this thesis. 

\item In Chapter 2, we begin with the motivation behind a search for neutral long lived particles predicted to exist is some BSM.
The physics of long-lived particle is also described.
This chapter also presents compelling  hints from theory, experiment as well as cosmological observation which point to the possible existence of NMLLPs.
An example BSM model used as guiding model to this analysis is also described leading to our event selection based of its predicted phenomenology. 
Previous analysis and their results are also shown. 

%experiment or LHC & CMS
\item  In Chapter 3,  describes the experimental setup particularly the LHC collider and CMS detectors and also a
detail description of the sub-detectors of the CMS which have been used in our search analysis.
%briefly presents the history of, and science behind, the subjects presented in this thesis.

%simulation or ECAL Timing Calibration
\item In Chapter 4, we provide a detail outline of Time measurement and Calibration of ECAL in order to 
reliably use timing as tool to search for long-lived particles which decay into photons and electrons depositing their energy unto the crystals of the ECAL.
%the experiment is outlined.
%reconstruction or Particle ID using ECAL Timing
\item Chapter 5,
The reconstruction of objects refered to as photons, jets and missing transverse energy ~\MET according the CMS standards is described here. The ECAL timing as useful for particle identification is presented.
%describes the simulation process used in the analysis.

% data_set  or Analysis Search Strategy
\item Chapter 6  hashes out our search strategy for neutral massive long-live particles. 
%follows the chain of reconstruction software used to obtain meaningful results from data.

%event_selection or Interpretation of results in GMSB
\item Chapter 7 provides an interpretation of the results obtained by our analysis  in the context of Gauge Mediated Supersymmetry Breaking(GMSB).
%hashes out the strategy for analysis and presents the data and simulated sets that will be used in the analysis.

%analysis or Future work
\item Chapter 8 outlines the possibility of future analysis strategy as we prepare to begin running LHC fully at $\sqrt{S} = 13$ \TeV in 2015. 
%7 demonstrates the implementation of the event selection processes.

%\item In Chapter 8 those events selected in Chapter 7 are analyzed.

%conclusion.
\item Chapter 9 outlines our discussion of the analyses presented and conclusions

\end{itemize}


\textbf{Some comments from Mississippi Snowmass 2013 I like.}
\paragraph*{Comments BSM?: Convener: Marcela Carena}
Energy Frontier
comments: Kyle Cranmar:
Questions: 
Comments: 
1)Measuring couplings of Higgs now @ LHC before ILC provide Full width.
2)No clear guidance at the moment to Empirical measurements reliable the most for now.
3) Smaller and shorter  time scale experiments better at the moment.
Comments: Andrei de Gouvan
1) Neutrino masses do not sit well in SM
2) Dark matter also not SM

Where to find new physics?
1) Neutrino, Dark Matter ? not easy
2) New Sources of CP -Invariance violation (EDMs) sensitive to very high energy scales
3)  Searches for violations
4) Precision measurements for Higgs Parameters and very rare processes?
5)  Explore LHC

Comments : Jonathan Feng:
1)  Neutrino Mass/theory of  Flavor - Neutrino Experiments
2) Higgs Boson and Naturalness( gauge Hierarchy problem- precision measurements of Higgs Boson parameters
3) Dark Matter -Experiments and search
4) Dark Energy and Modified gravity- Cosmological problem.
5) Matter Asymmetry - CP violations can be source to find Matter Asymmetry
6)  Inflation - Evidence for New physics can be found using -Experiments of CMB

Comments: Nima-Akarni Amed
1) Neutrino Oscillations + DM
2) Conservative Ideas still required to describe nature /observed data?
3) Effective Field theory of relativistic nature can also describe Condensed Matter physics, However, never has
a Scalar field been used as seen in the case of Higgs.
4) New Physics at TeV Scale making naturalness stable
5) Higgs Physics is fine-tuned thus needs precision measurements.
6) Is New physics as a result of naturalness or Not? - Test Naturalness with HL-LHC
7) HL-LHC killing naturalness? unlikely
8) If at O(100TeV) no New physics then fundamental shift paradigm in thinking
Expts: Proton decay,  DM sources, EDMs,  Build O(100 TeV) Machines

Comments: Regina
1) Explore unnatural areas to search for new physics by designing new Tools
2) Look back at clues to gain new Ideas for new measurements
3) Exploit existing facilities but also build new facilities.
4) Explore new territories-Using Neutrino Beams, Dark Matter, precision measurements
5) Explore Old ideas in the light of new technologies.
6) Redo already performed experiments using new technologies, data taking and Analyzing technologies.
Recent paper of Meta-stable particles
arXiv:hep-ph/0908.0315

Question? Why is the top quark mass so different from others: 
Answer in SUSY?.
%some CMS Bulletin Papers
Some impt CMS/ECAL papers to read.
1)CMSBUL-ARTICLE-2013-005
2) CMSBUL-ARTICLE-2013-023 %(http://cds.cern.ch/record/1561186?ln=en)
3) CMSBUL-ARTICLE-2013-012%(http://cds.cern.ch/record/1544496?ln=en)
4) ECAL Timing Prospects %(http://cms.cern.ch/iCMS/jsp/db_notes/showNoteDetails.jsp?noteID=CMS-20CR-2013/186)
5) 2013 ECAL Performancs NOTE: CMS DP -2013/016, CMS DP -2013/007 


%%%%%%%%%%%%%%%%%%%%%%%%%%%%%%%%%%%%%%%%%%%%%%%%%%%%%%%%%%%%%%%%%%%%%%%%%%%%%%%%
