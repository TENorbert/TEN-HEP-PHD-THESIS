%%%%%%%%%%%%%%%%%%%%%%%%%%%%%%%%%%%%%%%%%%%%%%%%%%%%%%%%%%%%%%%%%%%%%%%%%%%%%%%
% intro.tex: Introduction to the thesis
%%%%%%%%%%%%%%%%%%%%%%%%%%%%%%%%%%%%%%%%%%%%%%%%%%%%%%%%%%%%%%%%%%%%%%%%%%%%%%%%
\chapter{Introduction}
\label{intro_chapter}
%%%%%%%%%%%%%%%%%%%%%%%%%%%%%%%%%%%%%%%%%%%%%%%%%%%%%%%%%%%%%%%%%%%%%%%%%%%%%%%%
%The main style and messaging of my thesis will look like this:
%\begin{enumerate}
%\item Introduction as to why would anyone care about neutral long lived particles.
%\item What is/are the mechanisms behind a long lived particle? can you give an example and show this mechanism at work explicitly?
%\item How are searches for these particles performed and why use the LHC-CMS machines? eg Is there a particular reason why this is the case? Can these experiment be done using other machines?
%\item Have this searches been performed before? If yes, what are the results? What are their limits?  Can you beat their limits? What makes you think you can? What advantages or edge do you have over previous results?
%\item  How is the experiment performed? What specific issue(s) about the LHC-CMS machine makes it an ideal instrument?
%\item  What results did you get? Are they better than previous results? if not why? Can one improve on the results? What is the future of this kind of
%search? Can it be performed again  in future? What  advantages do future experiments have over this current experiments?
%\item  What are your thoughts(conclusion) about the whole process?
%\item  Where does one go from here?
%\end{enumerate}

%In this Thesis, I am going to write about three main aspects.
%\begin{enumerate}
%\item  Why long-lived particles  and the mechanism behind its possible existence?.
%\item  How to use the CMS detector to search for a long-lived particle.
%\item How is the search done?
%\item  How do I set an upper limit on the possible existence of a long-lived particle in case I fail to observe it?
%\item  What is the future for long-lived particle with the increase in LHC center of mass energy and luminosity?
%\end{enumerate}

\paragraph*{} \mbox{}\\
Many indirect observations at astronomical and cosmological scales\cite{DM}, indicate the presence of a new form of matter in the universe, which
only interacts significantly through gravity. This illusive matter which cannot emit or scatter electromagnetic radiation is call \textit{Dark Matter}~(DM).
The supposed existence of dark matter is one of the strongest indications for a new kind of physics beyond the standard model of particle physics, as no known particles can be attributed to dark matter. Visible matter make up only 4.5\% of the total matter in the universe and is best described using simple symmetries known as gauge symmetries. These gauge symmetries are expressed in a mathematical model called the \textit{Standard Model}~(SM). The SM describes with unmatched precision, the interactions and properties of the entire visible matter.  Despite the unprecedented success of the SM amongst scientific models, it cannot describe dark matter which make up the larger percentage of the matter in our universe. There have been numerous experimental reports on the discovery of particles predicted by the SM. On the other hand, there have been no reports on direct detection of the particles which make up dark matter. On the nature of DM, results from theoretical models and numerical N-body simulations \cite{DMS}, support speculation that dark matter could be made up of particles which are neutral and stable \ie have lifetime comparable to the age of our universe. There are also suggestions that dark matter particles could be produced from the disintegration~(decay) of other meta-stable particles called \textit{Long-Lived}~(LL) particles \cite{SUSYDM}. Long-lived particles can either be charged or neutral~(i.e cannot interact with light in the context of the SM). There is significant interest towards neutral dark matter particles,  since dark matter does not interact with light. A good amount of effort is directed towards the search for dark matter particles produced from the decay of \textit{neutral long-lived} particles\cite{LSPDM}. Experimental techniques for detecting neutral particles is limited and so new methods to search and detect dark matter particles are constantly being developed.  %Nevertheless, the ultimate goal is to detect dark matter particles!
%We believe DM consists of particles which are neutral, might be stable or meta-stable, very weakly interacting with ordinary matter and maybe massive.
\paragraph*{}%\mbox{}\\
The search for dark matter particles covers a wide range of experiments, from deep space search experiments like the Hubble and James Webb Space Telescope experiments, the Alpha Magnetic Spectrometer detector on board the International Space Station of NASA, ground based particle detector experiments like the Super Cryogenic Dark Matter Search experiment which try to detect dark matter particles produced in cosmic rays, to collider experiments like the Large Hadron Collider~(LHC) of the European National Laboratory for Nuclear Research in Geneva, Switzerland, where dark matter particles can be produced during particle collisions.
The interest in collider experiments is base on theoretical model predictions that dark matter particles or particles which decay into dark matter particles can be produced in a collider like the LHC, provided there is sufficient center-of-mass energy. These models are extensions of the standard model to allow for the existence of new fundamental particles. The models are called \textit{Beyond Standard Models}~(BSM) and there are several of them. \textit{Supersymmetry} is a BSM which extends the gauge symmetries to a much larger family of symmetries and allows for the doubling of the particles in the SM to include a wide variety of new fundamental particles which can be dark matter particles\cite{SUSYDM,LSPDM}. A particular branch of supersymmetry models called Gauge Mediating Supersymmetry Breaking~(GMSB) models, predict the production of new, massive, meta-stable, neutral, weakly interacting and long-lived particles which can decay into a candidate dark matter particle and isolated energetic photons. The candidate dark matter particle is stable and also neutral and weakly interacting. It has a lifetime comparable to the age of the universe. Detecting photons from the decay of a supersymmetry neutral and weakly interacting long-lived particle will be an indication of new particles, as there are no neutral, weakly interacting and long-lived particles which decay into photons in the SM. Numerous previous experiments have searched for these photons. Often, the results from these experiments have been negative. Nevertheless, new experiments with new detection techniques and clever search methods depending on the manner of production and decay of these new particles are continuously being developed. 
We will refer to such new, massive, weakly interacting and neutral long-lived particles which decay into isolated photons and candidate dark matter particles  as \textit{Neutral Massive Long-Lived Particles}~(NMLLP).
%Of particular interest to the scientific community are neutral LL particles, since DM is understood to not interact directly with light and could very weakly interact with visible matter.  Recent negative search results is indicating that dark matter particles, if they exists, could be very light i.e having very small mass of about a few eV to keV. These are known as Warm Dark Matter~(WDM). DM particles could also be heavy with mass in the \GeV to \TeV mass range. These are called Cold Dark Matter~(CDM). A common  property is that they are stable.
\paragraph*{}%\mbox{}\\
Our search analysis is motivated by the classic signature of a delayed photon from the decay of a NMLLP in GMSB models, produced in the LHC proton collider. The decay products of the NMLLP are detected using the multi-purpose Compact Muon Solenoid~(CMS) particle detector. We consider the NMLLP to be the Next-To-Lightest Supersymmetric Particle~(NLSP) called the \textit{lightest neutralino}~(\PSneutralinoOne). This lightest neutralino decays into a photon and the Lightest Supersymmetric Particle~(LSP) called the gravitino~($\tilde{G}$).  The gravitino is neutral and weakly interacting with visible matter, and being the LSP makes it stable. These properties makes the gravitino a good dark matter particle.. The neutralino decay can in general be instantaneous~(prompt) or \textit{delayed} depending on the choice of parameters in the neutralino decay model. In a subset of models called \textit{R-Parity Conserving}~(RPC) GMSB models, sypersymmetry particles are pair produced either, directly during particle collisions, or from the cascade decay of higher massive supersymmetry particles produced also in particle collisions. As a result, in these models, the LSP is very stable and automatically a dark matter candidate particle while the NLSP~(neutralino) is often long-lived and the photons are energetic, isolated and often delayed in their arrival time at a detector. These often delayed photons can be detected using the electromagnetic calorimeter~(ECAL) of the CMS detector.
The ECAL has an excellent timing resolution better than a \textit{nanosecond}~(ns). In cases where the neutralino is produced from the cascade decay of higher massive particles, large transverse momentum~(\pt) spray of hadronic particles collectively called jets are also part of the neutralino production and decay event. Since the gravitino is weakly interacting with the detector material, it is undetected. The gravitino presence is inferred using \textit{missing transverse momentum} which in combination with the transverse momenti of the jets and photons should conserve the total momentum of the event in the transverse plane of the detector.
In the decay scenario of the neutralino where its lifetime is large, say above, $3$~ns, the photon is delayed and its measured arrival time is large~(many nanoseconds~(ns)). This is because of the inherently long neutralino lifetime and the extra distance it has to travel inside the detector before is decays. 
Finding an event, with the combination of jets, missing transverse momentum and at least a delayed photon is a clear signal for new particled not known in the standard model~(SM).
The CMS detector is located in the LHC tunnel at one of the proton bunch crossing points also known as \textit{Interaction Point}~(IP).
We measure the arrival time of the photon from the IP to the surface of the ECAL. Relying on the excellent timing and energy resolution of the ECAL, we can distinguish between high energy photons from the decay of a NMLLP from those produced by the SM which are often prompt and not delayed. To ensure this excellent timing resolution, the ECAL detector is continuously time calibrated throughout the entire LHC proton-proton collision year.%to allow for optimum performance in its timing measurements. 
 We will describe in future chapters how  the ECAL timing alignment is done to realize this excellent timing resolution.
Observation of a delayed photon event at the LHC using ECAL timing measurements, will confirm the existence of NMLLPs and will help answer important questions in particle physics like: What is the source and nature of dark matter?, Is there any reason why known SM particles are classified into 3 generations and have very different masses as known? Why do we observe so much matter compared to anti-matter in our universe? Is there a single universe or multiverses?
Answers to these questions will provide a clear understanding and direction for future research in physics beyond the SM.
\paragraph*{} %\mbox{}\\
In this thesis, we have described our search for a NMLLP decaying to a photon using arrival time information of the photon to the ECAL and use this information to distinguish between the signal from new long-lived particles and background from SM interactions. Our search analysis uses data recorded using the CMS detector produced from proton-proton~(\textbf{$pp$}) collisions at the Large Hadron Collider~(LHC) with a center-of-mass energy  $\sqrt{S} = 8 $\TeV.
The contents of this thesis is arranged beginning with this introduction as chapter 1, followed by chapter 2, which gives a 
brief description of the SM, highlighting its strengths and weaknesses which motivates why we need to go beyond the SM in our efforts to understanding the universe. The study \textit{supersymmetry} as our BSM physics model, paying particular attention to GMSB models which allows for the existence of NMLLP which can decay into a photon and a gravitino with jets in the associate event.
In Chapter 3,  we describe the  LHC and CMS particle detector, dueling only on the sub-detectors which have been used in our search analysis. How timing measurements of a partile are made by the ECAL sub-detector is described in chapter 4.
The reconstruction of a full event with its constituent particles is described in chapter 5 with the definition of quantities like jets and  missing transverse energy~(\MET) according the CMS standards is described. Anomalous signals called \textit{spikes} observed in the  ECAL is mentioned.
% Search Analysis for LL particles 
Our search strategy is described in chapter 6 with details of which datasets and triggers we have used, what is trigger efficiency, our choice of search observable, event selection and background estimation techniques used. We describe the various systematic sources considered and their contribution to our search result is presented in this chapter.
%Our search result ed here. Sources of systematic and quantification considered in this experiment are also described.
%Statistical Analysis
Chapter 7 presents the statistical and analysis methods used with clear meaning of $p$-values as used in our analysis.
%hashes out the strategy for analysis and presents the data and simulated sets that will be used in the analysis.
%Limit Interpretation & Future
The cross section times branching ratio limits depending on the lifetime and mass of the production and decay of a NMLLP using the minimal \textit{Gauge Mediating Supersymetric Model}, with \textit{Snowmass Signal Point} 8~(SPS8) as our benchmark model is presented in chapter 8. The interpretation of our results in terms of exclusion regions reached by our analysis is also presented in this chapter. %Possible improvement for future analysis is briefly mentioned. 
%Conclusion.
%\item  
Chapter 9 covers the conclusion.



%\textbf{Some comments from Mississippi Snowmass 2013 I like.}
%\paragraph*{Comments BSM?: Convener: Marcela Carena}
%Energy Frontier
%comments: Kyle Cranmar:
%Questions: 
%Comments: 
%1)Measuring couplings of Higgs now @ LHC before ILC provide Full width.
%2)No clear guidance at the moment to Empirical measurements reliable the most for now.
%3) Smaller and shorter  time scale experiments better at the moment.
%Comments: Andrei de Gouvan
%1) Neutrino masses do not sit well in SM
%2) Dark matter also not SM

%Where to find new physics?
%1) Neutrino, Dark Matter ? not easy
%2) New Sources of CP -Invariance violation (EDMs) sensitive to very high energy scales
%3)  Searches for violations
%4) Precision measurements for Higgs Parameters and very rare processes?
%5)  Explore LHC

%Comments : Jonathan Feng:
%1)  Neutrino Mass/theory of  Flavor - Neutrino Experiments
%2) Higgs Boson and Naturalness( gauge Hierarchy problem- precision measurements of Higgs Boson parameters
%3) Dark Matter -Experiments and search
%4) Dark Energy and Modified gravity- Cosmological problem.
%5) Matter Asymmetry - CP violations can be source to find Matter Asymmetry
%6)  Inflation - Evidence for New physics can be found using -Experiments of CMB

%Comments: Nima-Akarni Amed
%1) Neutrino Oscillations + DM
%2) Conservative Ideas still required to describe nature /observed data?
%3) Effective Field theory of relativistic nature can also describe Condensed Matter physics, However, never has
%a Scalar field been used as seen in the case of Higgs.
%4) New Physics at TeV Scale making naturalness stable
%5) Higgs Physics is fine-tuned thus needs precision measurements.
%6) Is New physics as a result of naturalness or Not? - Test Naturalness with HL-LHC
%7) HL-LHC killing naturalness? unlikely
%8) If at O(100TeV) no New physics then fundamental shift paradigm in thinking
%Expts: Proton decay,  DM sources, EDMs,  Build O(100 TeV) Machines

%1) Explore unnatural areas to search for new physics by designing new Tools
%2) Look back at clues to gain new Ideas for new measurements
%3) Exploit existing facilities but also build new facilities.
%4) Explore new territories-Using Neutrino Beams, Dark Matter, precision measurements
%5) Explore Old ideas in the light of new technologies.
%6) Redo already performed experiments using new technologies, data taking and Analyzing technologies.
%Recent paper of Meta-stable particles
%arXiv:hep-ph/0908.0315

%Question? Why is the top quark mass so different from others: 
%Answer in SUSY?.
%some CMS Bulletin Papers
%Some impt CMS/ECAL papers to read.
%1)CMSBUL-ARTICLE-2013-005
%2) CMSBUL-ARTICLE-2013-023 %(http://cds.cern.ch/record/1561186?ln=en)
%3) CMSBUL-ARTICLE-2013-012%(http://cds.cern.ch/record/1544496?ln=en)
%4) ECAL Timing Prospects %(http://cms.cern.ch/iCMS/jsp/db_notes/showNoteDetails.jsp?noteID=CMS-20CR-2013/186)
%5) 2013 ECAL Performancs NOTE: CMS DP -2013/016, CMS DP -2013/007 


%%%%%%%%%%%%%%%%%%%%%%%%%%%%%%%%%%%%%%%%%%%%%%%%%%%%%%%%%%%%%%%%%%%%%%%%%%%%%%%%
